%%%%%%%%%%%%%%%%%%%
%% imports
%%%%%%%%%%%%%%%%%%%
\documentclass[12pt,twoside,a4paper,parskip]{scrbook} % oneside for writing, twoside for submission (space for comments on the print)
\setcounter{tocdepth}{2} % depth of table of contents, for debugging purposes
\usepackage[utf8]{inputenc} % define the encoding type for LateX, may be obsolete in newer versions of LateX
\usepackage{csquotes} % helper package for quotes
\usepackage[english]{babel} % language support, multiple languages can also be listed
\usepackage{mathptmx} % Sets text and math to Times-like font
\usepackage{floatflt} % wrap text around tables and figures, useful for some templates
\usepackage[pdftex]{graphicx} % /includegraphics command with more options
\usepackage[hidelinks]{hyperref} % used for cross-referencing in LateX, hidelinks is an option to better format links
\usepackage{xcolor} % color management
\usepackage{amssymb} % symbol support
\usepackage{textcomp} % text symbol support
\usepackage{nicefrac} % better formatting for fractions
\usepackage{pdfpages} % include pdf pages in LaTeX
\usepackage{float} % floating objects (i.e. figures, tables)
\usepackage{pdflscape} % better pdf support with more options
\usepackage[verbose]{placeins} % checks that floats are within bounds - verbose adds extra logs
\usepackage[markcase=ignoreuppercase,headsepline,plainfootsepline]{scrlayer-scrpage} % manage page styles, markcase: automatic typesetting in heads, headsepline: line underneath the headerm, plainfootsepline: rule above the footer
\usepackage[ruled,vlined]{algorithm2e} % algorithm environment, ruled: algorithm with line at the top and bottom. Caption is not centered but at the beginning of the algorithm, vlined: vertical line followed by a small horizontal line between the start and the end of each block
\usepackage{caption} % adds captions to figures, tables, equations... with correct number assignments
\usepackage{subcaption} % adds subcaptions to subfigures, useful for more complex figures, has to be imported after the caption package
\usepackage{epstopdf} % needed to use .eps 
\usepackage{longtable} % enables long tables over page boundaries
\usepackage{setspace} % sets line spacing
\usepackage{booktabs} % better line management and line separation within tables
\usepackage[sortcites,style=numeric,backend=biber,doi=false,isbn=false,url=false,eprint=false]{biblatex} % good package for bibliography support in LaTeX with biber backend, sortcites: sorting bibliography (name, title, year), style=numeric: standard numeric citation style (recommended!), doi/isbn/url/eprint=false: hide links in bibliography (recommended for clean bibliography!)
\usepackage{makecell} % tabular column heads and multilined cells
\usepackage{amsmath} % math support 
\usepackage{scrhack}

\bibliography{bibliography} % included biblatex formatted citations from the file bibliography.bib

\usepackage{lipsum} % only used for the template generation, remove this usepackage command and all \lipsum calls

\usepackage{booktabs} % For professional looking tables
\usepackage{tabularx} % For auto-adjusting column widths
\usepackage{geometry} % Adjust page margins if necessary
\geometry{margin=1in}

\usepackage{graphicx} % Add this in the preamble if not already included
\usepackage{booktabs}
\usepackage{multirow}
\usepackage{float}



%%%%%%%%%%%%%%%%%%%
%% definitions
%%%%%%%%%%%%%%%%%%%
\def\BaAuthor{Bibin Babu}
\def\BaAuthorStudyProgram{MAI} 
\def\BaType{Master Thesis}
\def\BaTitle{Determination of  Drug Efficacy  
on Pancreatic Tumor 3D Spheroidal Tissues}
\def\BaSupervisorOne{Prof. Dr. Magda Gregorová}
\def\BaSupervisorTwo{Prof. Dr. Jan Hansmann}
\def\BaDeadline{\SubmitDate}
\def\SubmitDate{January 15th 2025}
\def\git{https://github.com/.../...}
\def\iswithfullname{}  % Define this to indicate that the full name should be shown

% option to generate anonymous submission for plagiarism scan!
\ifdefined\iswithfullname
\def\ShowBaAuthor{\BaAuthor}
\else
\def\ShowBaAuthor{N.~N.}
\fi

\hypersetup{
pdfauthor={\ShowBaAuthor},
pdftitle={\BaTitle},
pdfsubject={Subject},
pdfkeywords={Keywords}
}

%%%%%%%%%%%%%%%%%%%
%% configs to include
%%%%%%%%%%%%%%%%%%%
\colorlet{punct}{red!60!black}
\definecolor{background}{HTML}{EEEEEE}
\definecolor{delim}{RGB}{20,105,176}
\colorlet{numb}{magenta!60!black}

\definecolor{gray}{rgb}{0.4,0.4,0.4}
\definecolor{darkblue}{rgb}{0.0,0.0,0.6}
\definecolor{cyan}{rgb}{0.0,0.6,0.6}

\definecolor{pblue}{rgb}{0.13,0.13,1}
\definecolor{pgreen}{rgb}{0,0.5,0}
\definecolor{pred}{rgb}{0.9,0,0}
\definecolor{pgrey}{rgb}{0.46,0.45,0.48}

\newcommand*{\forcetwoside}[1][1]{%
 \begingroup
   \cleardoubleoddpage
   \KOMAoptions{titlepage=true}% useful e.g. for scrartcl
   \csname @twosidetrue\endcsname
   \maketitle[{#1}]
 \endgroup
}
\def\c#1{\mathcal{#1}}
\DeclareUnicodeCharacter{2212}{-}

\begin{document}

%%%%%%%%%%%%%%%%%%%
%% Title page
%%%%%%%%%%%%%%%%%%%
\ifdefined\print
\newgeometry{centering}    %%% make the page centered on paper
\begin{titlepage}
	\newcommand{\HRule}{\rule{\linewidth}{0.5mm}}
	\centering
	\begin{center}
		\vspace*{.06\textheight}	
		\HRule \\[0.4cm] % Horizontal line
		{\huge \bfseries \BaTitle \par}\vspace{0.4cm} % Thesis title
		\HRule \\[1.5cm] % Horizontal line
		\vspace{4.5cm}
		\begin{minipage}[t]{0.4\textwidth}

		\centering
		{\large \BaAuthor} 

		\end{minipage}
		\vfill
		\vspace*{.1\textheight}
		{\large \BaType}\\[0.5cm] % Thesis type
		{\large University of Applied Sciences Würzburg-Schweinfurt}\\[0.5cm]
		{\large \SubmitDate}\\[4cm] % Date
		\vfill
	\end{center}
\end{titlepage}
\restoregeometry
\fi

\frontmatter
\titlehead{%  {\centering side head}
  {Technical University of Applied Sciences Würzburg-Schweinfurt (THWS)\\
   Faculty of Computer Science and Business Information Systems\\\\
	}}
\subject{\BaType}
\title{\BaTitle\\[15mm]}
\subtitle{\normalsize{submitted to the Technical University of Applied Sciences W\"{u}rzburg-Schweinfurt in the Faculty of Computer Science and Business Information Systems to complete a course of studies in \BaAuthorStudyProgram}}
\author{\ShowBaAuthor}
\date{\normalsize{Submitted on: \BaDeadline}}
\publishers{
  \normalsize{Initial examiner: \BaSupervisorOne}\\
  \normalsize{Secondary examiner: \BaSupervisorTwo}\\
}
\lowertitleback{
\centering\includegraphics[width=4cm]{figures/qrcode-thesis.png}
}
\forcetwoside

%%%%%%%%%%%%%%%%%%%
%% abstract
%%%%%%%%%%%%%%%%%%%
\section*{Abstract (en)}
Pancreatic tumor treatment is hindered by the intricate nature of tumors and their diverse microenvironments. This complexity necessitates an exploration into identifying optimal drug combinations and concentrations tailored to each patient's specific tumor characteristics. This thesis aims to assess drug efficacy by ranking these various drug combinations and concentrations. The ranking is based on features extracted from bright-field microscopy images of three-dimensional tumor tissue models using representation learning. The core challenge is to learn robust features that accurately characterize alterations in these tumor tissue models induced by drug application over time. This research seeks to develop a standardized and effective approach for evaluating drug efficacy, potentially improving treatment outcomes for pancreatic tumor patients.

% cSpell:disable
\section*{Abstract (de)}
Die Behandlung von Bauchspeicheldrüsentumoren wird durch die komplexe Natur der Tumore und ihre vielfältigen Mikroumgebungen behindert. Diese Komplexität erfordert eine Untersuchung zur Identifizierung optimaler Medikamentenkombinationen und -konzentrationen, die auf die spezifischen Tumoreigenschaften jedes Patienten zugeschnitten sind. Diese Masterarbeit zielt darauf ab, die Wirksamkeit von Medikamenten zu bewerten, indem sie diese verschiedenen Medikamentenkombinationen und -konzentrationen einstuft. Die Bewertung basiert auf Merkmalen, die aus Helligkeitsmikroskopiebildern dreidimensionaler Tumorgewebsmodelle mittels Repräsentationslernen extrahiert werden. Die zentrale Herausforderung besteht darin, robuste Merkmale zu erlernen, die Veränderungen in diesen Tumorgewebsmodellen genau charakterisieren, die durch die Anwendung von Medikamenten über Zeit induziert werden. Diese Forschung zielt darauf ab, einen standardisierten und effektiven Ansatz zur Bewertung der Medikamenteneffizienz zu entwickeln, der möglicherweise die Behandlungsergebnisse für Patienten mit Bauchspeicheldrüsentumoren verbessert.

\newpage
\chapter*{Acknowledgment}
\vspace{1em}

\setstretch{1.5}
Thank all who believed in me.

Thank Prof. Magda for guiding me to how to think scientifically and also for the SimCLR and autoencoder prediction ranking startegy.

Thank Prof. Jan for the softmax approach idea.

Thank Dalia for the code for sharp layer calculation and thanks for the biological stuff explanation and the once in a time opportunity.

Thank Phillip for the cell viability test idea.

Thank stefan for the microscope focal length explanation.

Thank Philipp for the resize and crop augmentationc idea and general explanation for th edata augemnetation.

Thank Dad and Mom for this life. Without you guys I won't be able to write this.

Danke.
\vspace{2em}

\hfill Würzburg, on 15.01.2025

\vspace{2em}

\noindent\rule{6cm}{0.4pt}

\vspace{0.5em}

\noindent Bibin Babu
%%%%%%%%%%%%%%%%%%%
%% toc
%%%%%%%%%%%%%%%%%%%
\tableofcontents

%%%%%%%%%%%%%%%%%%%
%% List of Figures and Tables
%%%%%%%%%%%%%%%%%%%
\listoffigures
\addcontentsline{toc}{chapter}{List of Figures}

\listoftables
\addcontentsline{toc}{chapter}{List of Tables}

%%%%%%%%%%%%%%%%%%%
%% Main part of the thesis
%%%%%%%%%%%%%%%%%%%
\mainmatter

% Include chapters
\chapter{Introduction}\label{ch:intro}

Pancreatic tumor presents a significant challenge in terms of treatment due to its heterogeneous nature and the mutations that occur during its progression
 within the human body. Clinicians rely on case studies, human trials, and their own expertise gained from past patient treatments to select drugs for new 
 patients. However, this approach is often based on trial and error, with varying outcomes. Patients may experience either successful treatment or severe 
 side effects such as hair loss and damage to other organs. Since each patient's tumor cells exhibit unique characteristics influenced by factors such as 
 age and genetics, treatments that have worked for one patient may not be effective for another. Consequently, clinicians may need to change the prescribed drugs or try different combinations, which can lead to delays and increased risks for the patient, including mortality.

In light of these challenges researchers at Fraunhofer Translational Center for Regenerative Therapy TLZ-RT Wuerzburg,  propose a vision for the future: 
cultivating multiple three-dimensional tumor tissue models for each patients in the lab using biopsy samples and studying the efficacy of drugs on these 
three-dimensional tumor tissue models first.\textit{(Note: In this thesis, "3D tumor tissue models or tumor tissue models" refers to physical, lab-grown
 tissues and not computational or AI models.)} By conducting drug development experiments and analyses on these tissue models, they aim to find the optimum 
 or best drug combination tailored to each patient's specific tumor characteristics.This approach can minimize direct side effects on human patients and 
 reduce the time needed to select the most effective personalized treatment, thereby decreasing the risk of that patient's mortality. Additionally, it can 
 significantly reduce the cost and time of preclinical testing in the drug development process. Ultimately, these information obtained from drug efficacy 
  assessment experiments can inform clinicians' decisions, enabling them to select the most effective drug combination before administering it to the patient.

As a proof of concept, The Fraunhofer TLZ-RT Würzburg laboratory utilizes a modular dual-arm robot-based system \cite{Dembski2023Establishing}, equipped 
with incubators and bioreactors (see Figure \ref{fig:platform} and Figure \ref{fig:Robot}) under physiological conditions to study drug efficacy for the
 long-term culture of these three-dimensional tumor tissue models. One advantage of this platform is its ability to capture bright-field microscopy images 
 (detailed explanation in Chapter \ref{ch: DataDescription}) of 3D tumor tissue models using a customized microscope setup integrated into a robotic 
 platform, providing flexibility in image acquisition to meet experimental requirements.

\begin{figure}[H]
  \centering
  \includegraphics[scale=0.4]{figures/platform setup.png} % Adjust the width as needed
  \caption{Robo platform}
  \label{fig:platform}
\end{figure}



\begin{figure}[H]
  \centering
  \includegraphics[scale=0.4]{figures/robot.png} % Adjust the scale factor as needed
  \caption{Dual-arm robot}
  \label{fig:Robot}
\end{figure}

Although the vision for the future is to simulate the identical interaction environment of drugs with tumor cells as it occurs in the human body, current technology has not yet achieved this. The current three-dimensional tumor tissue models developed in the lab do not fully resemble real pancreatic tumor cells found in the human body. These 3D tumor tissue models only contain pure tumor tissues, whereas real human pancreatic tumor cells exist within a complex microenvironment comprising tumor cells, blood vessels, other tissues, and various cell types.
Fortunately, if human body tumor cells can be replicated in the lab in the future, the techniques currently used to study bright-field microscopy images will still be applicable. However, the fact that bright-field microscopy images are two-dimensional limits the ability to perform a comprehensive analysis of the drug's impact on the entire 3D structure of the cultivated tumor tissue models. Despite this limitation, this research serves as a valuable starting point for studying drug efficacy in a controlled environment. 

Alternatives to bright-field microscopy images include 3D fluorescence microscopy and luminescent cytotoxicity assays. However, both methods are invasive. Fluorescent molecules tend to generate reactive chemical species under illumination, enhancing phototoxic effects. This chemical reaction with the 3D tumor tissue model may alter its structure, making it not suitable to isolate the drug's effect over time. Similarly, luminescent cytotoxicity assays result in a dead culture, rendering them unsuitable for longitudinal studies. Additionally, both methods require removing the well plate from the isolated culture environment for extended periods, making the samples susceptible to external environmental factors. For instance, in fluorescence microscopy, cells are particularly vulnerable to phototoxicity from short wavelength light. In contrast, bright-field microscopy images are non-invasive, allowing continuous culture and the possibility of creating time series of images to study dynamic changes. Therefore, we rely on bright-field microscopy images to study the time-evolutionary effects of drugs.

\section{Laboratory Setup}
\label{sec:lab-setup}
3D tumor tissue models are cultured in well plates containing 96 wells, each providing a nutrient medium that allows them to maintain their tissue-specific functions in vitro. Although each plate can yield 96 pure 3D tumor tissue models, the edge effect is accounted for, where outer wells may be exposed to variable conditions such as temperature fluctuations, increased evaporation rates, and other environmental factors. Consequently, we restrict our analysis to the 60 inner wells per plate as in figure \ref{fig:Wellplate}, adhering to standard procedures to ensure consistent and reliable experimental data. 

\begin{figure}[H]
  \centering
  \includegraphics[width=0.9\linewidth]{figures/WellPlate.png} % Adjust the width as needed
  \caption{A well plate containing 96 wells where rows A, H and columns 1, 2 are excluded due to edge effects.}
  \label{fig:Wellplate}
\end{figure}

Based on the drug concentration applied to 3D tumor tissue models, the bright-field microscopy images we capture can be categorized into three:

\begin{enumerate}
  \item Control (0 percentage drug applied) For easiness, we refer to this category as ``Untreated''
    
  \item Single concentration (theoretically recommended single concentration of drug treatment) For easiness, we refer to this category as ``Single dose''
        


  \item Control Day 10: The untreated images show a morphological change where the center of the tumor tissue becomes darker, resembling the darkness
   observed when drugs are applied on Day 10. This occurs due to the natural death of tumor cells within the tissue, caused by a lack of nutrients to 
   sustain the cancer cells.
   \item Drug Screening: Different drug combinations and concentrations are used in this experimental study to evaluate drug efficacy. Some images may 
   resemble single-dose images, while others may resemble cond10 if the drug is ineffective. Additionally, certain drugs may or may not result 
   in the destruction of surrounding non-tumor cells in the human body, potentially causing side effects. For simplicity, we refer to this category as 
   "Drug Screening." These are the images that require relative assessment through ranking.
        
\end{enumerate}

\textcolor{red}{explain here what is explod }
\begin{figure}[H]
  \centering
  \includegraphics[scale=0.6]{figures/types.png} 
  \caption{Three different types of images: Drug Screened, Single Dose, and Untreated as mentioned in section ~\ref{sec:lab-setup}.}
  \label{fig:originals}
\end{figure}

We apply single dose and drug screened combinations in different well plate settings as illustrated inFigure \ref{fig:Single dose wellplate} and figure 
\ref{fig:Drug screen wellplate}  

\begin{figure}[H]
  \centering
  \includegraphics[width=0.9\linewidth]{figures/singledose.png} % Adjust the width as needed
  \caption{Well plate setup for the single-dose experiment where the left half remains untreated and the right half is treated with a single drug 
  concentration. This image was taken three days after drug application.}
  \label{fig:Single dose wellplate}
\end{figure}

\begin{figure}[H]
  \centering
  \includegraphics[width=1\linewidth]{figures/drug_screened.png} % Adjust the width as needed
  \caption{Well plate setup for the drug screening experiment where the majority of tumor tissues are treated with different combinations of drug 
  concentrations (multi-colored wells), while some are left untreated (white wells bounded by orange box).}
  \label{fig:Drug screen wellplate}
\end{figure}

\ref{fig:time} Illustrates the flow chart of time evolutoion of 3D tumor tissues.
The 3D tumor tissue models develop in the wellplate progressively from day 1 reaching their maximum cancerous state by day 7, at which point the drug is 
administered. By day 10, the drug's effect on the cancerous tissue is expected to peak as nutrient availability gradually decreases, causing the tumor to 
diminish. Therefore, to isolate the drug's effects, changes in tumor tissue deterioration are assessed on the peak effect day, i.e., day 10, in 
accordance with established medical protocols and previous research findings.

\begin{figure}[H]
  \centering
  \includegraphics[scale=0.7]{figures/timed.png} 
  \caption{Illustrates the flow chart of time evolutoion of 3D tumor tissues.}
  \label{fig:time}
\end{figure}
\let\cleardoublepage\clearpage


\chapter{Motivation} \label{chapter:Motivation} 

We assess the efficacy of the drug by comparing the changes it induces in the bright-field microscopy images over a period of time. 
The current methods to differentiate these changes involve studying the alterations from day 7 to day 10. These changes are typically
observed in three main parameters: 

\begin{enumerate}
  \item Size/Area
  \item Circularity/Diameter/Perimeter
  \item Pixel intensity or color change
\end{enumerate}

These parameters serve as human-interpretable metrics for assessing the efficacy of the drug. However, these manual methods are inherently limited to
observable features, which may overlook subtle or complex patterns within the data.


Representation learning techniques, such as those based on deep learning, can be employed to extract features that are not immediately interpretable by 
humans. These methods leverage neural networks to learn high-dimensional feature representations directly from the images, capturing intricate patterns, 
relationships, and variations that may correspond to biological phenomena. Furthermore, these learned representations enable a standardized approach to analysis. Unlike manual assessments, which can vary due to human subjectivity,
representation learning models produce consistent results once trained. 

This is the focus of my master thesis, where I take on the crucial role of developing and applying representation learning techniques to uncover these
hidden patterns and standardize the analysis process.

\let\cleardoublepage\clearpage


\chapter{Research questions}\label{ch:Research questions}
This thesis aims to assess drug efficacy by ranking different drug combinations and concentrations. 
The ranking is based on features extracted from bright-field microscopy images of  three-dimensional tumor tissue models using
representation learning. 

1. Can we learn latent features that capture the alterations 
induced in three-dimensional tumor tissue models by drug application over a period of time from bright-field microscopy images?

2.  Will these features can effectively establish a ranking of drug efficacy?

3. What methodologies and frameworks can be employed to extract and learn these
 hidden representations efficiently?

4. What could be reasonable metrics, such as L2 loss or cosine similarity, for 
supporting the relative assessment of drug efficacy?

5. can unsupervised learning be used to tackle this ranking problem since lack of label is 
our main challenge?

6. How do we deal with the position change in the image collection of day 10 if we want to use prediction 
model as one of our ranking startegy?

7. How do we incorporate the fact about brightness/blur change in the image collection due to environmental or microscope 
variations?

8. how do we deal with the insufficient data problem?

9. Do we actually need to learn these latent features to establish effective ranking ? or do would it be possible to do it directly using images it self?

10. If we use simclr then is our dataset sensitive to strong data augementation in that case if we reduce the intensity will it able to learn the features?

11. What to do if same concentraion drug have diferent effect on the cancer cell in different experimant? ( we don't know if its visual effect or real efficay effect)
but for sure there is visual effect in the sense debris. 

12. Does medical gray scale image will give better performance with  strong data augementation or will it be worse since its sensitive to original distribution.
13. Does Data augmentation works as a learning tool to learn the invariant features or it is act as increasing the sample size so that it sees all possible distribution
 that can happen to the same image groups?

 

\chapter{Literature Review}\label{ch:Literature Review}

\textbf{Base neural network architecture for representation learning.} Learning visual representations of medical images, such as X-rays (radiographic images) and bright-field microscopy images, is crucial for medical image understanding. However, progress in this area has been hindered by the heterogeneity and complexity of subtle features in these images, especially when they don't have labels. Existing work often relies on fine-tuning weights transferred from ImageNet pretraining (Wang et al., 2017 \cite{8099852} ; Esteva et al., 2017 \cite{Esteva2017Dermatologist} ; Irvin et al., 2019 \cite{irvin2019chexpert} ), which is suboptimal due to the drastically different characteristics of medical images. 

To address these challenges, researchers have proposed various innovative approaches. ConVIRT \cite{zhang2022contrastive} offers an alternative unsupervised strategy for learning medical visual representations by exploiting naturally occurring paired descriptive text. This method introduces a new approach to pretraining medical image encoders using paired text data via a bidirectional contrastive objective between the two modalities. It is domain-agnostic and requires no additional expert input.  However, given the absence of specific paired text data for our image dataset, ConVIRT does not offer a solution tailored to our specific problem.

The contrastive loss used in ConVIRT is derived from the SimCLR \cite{chen2020simple} self supervised learning framework. SimCLR learns representations by maximizing agreement between differently augmented views of the same data example via a contrastive loss in the latent space. The framework consists of a neural network base encoder that extracts representation vectors from augmented data examples. The framework allows for various choices of network architecture without any constraints. The authors opt for simplicity and adopt ResNet \cite{he2015deepresiduallearningimage}, introducing a learnable nonlinear transformation between the representation and the contrastive loss to substantially improve the quality of the learned representations. However, these methods require careful treatment of negative pairs, typically relying on large batch sizes to retrieve them. Additionally, their performance is highly dependent on the choice of image augmentations. 

Once using the ranking strategies explained in the chapter methodologies for ranking, then what could be reasonable metrics, such as L2 loss or cosine similarity, to support the relative assessment of drug efficacy was the next question to tackle the raltive positioning of dru screening images to 3 classes we selected as we explained in chapter ranking. Cosine distance was the first choice since simclr loss designed to have higher similarity between similar images. This paper \cite{Mualla2013ACD} used euclidean distance to find the nearby cells for unstained cell detection in bright-field microscope images. Notably, alternative metrics have proven useful in related applications; for example, Author of the paper \cite{CrossZamirski2022LabelFree} used Pearson correlation coefficient to measure similarity between the predicted images (or their extracted features) and the ground truth images for the application of label-free Cell Painting by predicting the five fluorescent Cell Painting channels from brightfield input.  In the study \cite{Lee2018DeepHCS}   DeepHCS: Bright-Field to Fluorescence Microscopy Image Conversion, L1 distance (Mean Absolute Error, MAE) was used as a loss function to measure the pixel-wise error between predicted and ground truth fluorescence images. This approach ensures accurate translation of brightfield images into synthetic fluorescence images. The aforementioned study \cite{Todorov2023STrack} on bacterial cell tracking (Assessing the overlap of cell regions between consecutive frames) used the Jaccard index to compare the performance of different tracking tools, including on brightfield images of P. protegens and S. pneumoniae. These findings collectively inform to use Cosine distance, euclidean distance, Pearson correlation coefficient, L1 distance, and Jaccard index as reasonable metrics to support the relative assessment of drug efficacy.

Since lack of data is a problem one of the most important problem, we need to explore rannking srategy that doesn't need specific training, which is why we choosed dimenstionaily reduction technics like PCA, t-SNE, UMAP. \cite{keyes2020cancerprimer} paper applied PCA, t-SNE, UMAP to high-dimensional cytometry datasets in the study of human cancer, specifically to analyze bone marrow aspirates from acute myeloid leukemia patients to reduce it to two or three dimensions for analyzing and interpreting cytometry data. 

In another study \cite{melba:2024:020:woodland}, researchers applied PCA, UMAP, and t-SNE to reduce the dimensionality of bottleneck features from a Swin UNETR trained for liver segmentation on T1-weighted MRIs. They found that either PCA or UMAP improved performance over average pooling for all models test. Since there is studies that shows PCA; Tsne and umap works well for medical images, we further explore hem to reduce the higher dimensionality of our image data to single dimension and use them for ranking. 
\chapter{Data Description}\label{ch: DataDescription}
\section{Data set}
\label{sec:Data set}
As explained in the chapter \ref{ch:intro}, the dataset consists of images taken by bright-field microscopy, which operate as follows: 

The tumor 3D spheroidal tissue model is illuminated by transmitted white light (i.e, light is projected from below and observed from above).
 The contrast in the image is created due to the reduction in light intensity as it passes through denser regions of the sample, where more light is
  absorbed or scattered. This results in a typical bright-field microscopy image where the sample appears darker against a bright background, giving 
the technique its name. In our case, the 3D tumor model appears as a dark grayish structure on a bright background. 
For simplicity, bright-field microscopy images will be referred to as 'images'.

 Since the tissue models are cultivated using robotic arms, the process sometimes fails to replicate the natural shapes and patterns that typically occur when laboratory personnel manually cultivate the samples. To ensure the 
 dataset's quality and consistency, Dalia Mahdy, Phd student who works in the lab, pre-processed the images through a machine learning model to filter out invalid ones. By 'invalid images' , we mean
  those that variates from the expected morphology of the tumor tissues as cultivated by lab personnel. These images are typically elongated either in height or width as shown in second image 
 of figure \ref{fig:valid}. I collected these filtered images via USB and Google Drive in their original TIFF format. 

  \begin{figure}[H]
    \centering
    \includegraphics[scale=0.25]{figures/finevalid.png} 
    \caption{Examples of bright-field microscopy images with valid and invalid (unexpectedly elongated) morphology of tumor tissue models.}
    \label{fig:valid}
  \end{figure}

  The original images are 2456x2054 pixels in size, in 16-bit grayscale, and consist of multiple layers. These layers are obtained by capturing images
   at different focal planes in brightfield microscopy. The number of layers can vary, as images can be taken at any number of focal planes. The sharpness 
   of each layer of an image depends on how well the focal plane of the microscope aligns with the depth of the tumor tissue model. Only one layer of this 3D tumor tissue model will 
   be perfectly in focus, while others may appear blurry because they are slightly above or below the focal point as you can see in figure \ref{fig:blur}. Combining these focal planes later in 
   computational analysis can provide richer data, even if some layers are blurry individually. This is one of the reasons why I decided to use multiple
    layers. The images I received from the lab mostly have three layers, while a few have one or five layers.

    \begin{figure}[H]
      \centering
      \includegraphics[scale=0.4]{figures/blur.png} 
      \caption{Different layers of the same image reveal that, from A to C, the region above the white dotted line becomes sharper, while the region below becomes blurry }
      \label{fig:blur}
    \end{figure}

    Within the same experiment, images are 
    captured using consistent acquisition settings. that means each image in the same experiment consists of different layers captured at same focal lengths. For example, if image 1 has layers corresponding to focal
 lengths A, B, and C, Image 2 will also have layers corresponding to the same focal lengths A, B, and C.  However, for images from different experiments, the focal lengths may differ. For instance, images from 
 experiment 1 (single dose) may have slightly different acquisition settings compared to images from experiment 2 (drug screening), leading to variations 
 in the focal lengths and corresponding layers. 
 Secondly, the layers in each image are slightly misaligned when stacked, as shown in the figure  \ref{fig:Misalignment}. This misalignment of layers within the image and variation in focal length inbetween the images may or may not affect the performance of ranking.

 \begin{figure}[H]
  \centering
  \includegraphics[scale=0.25]{figures/posi.png} 
  \caption{A small section of the image was zoomed in and separated to show the misalignment of the three stacked layers.}
  \label{fig:Misalignment}
\end{figure}
\begin{figure}[H]
  \centering
  \includegraphics[scale=0.2]{figures/noise.png} 
  \caption{Noise}
  \label{fig:noise}
\end{figure}
We observed some temporal noise/artifacts, as shown in figure  \ref{fig:noise} in images. The presence of artifacts in the well-plates could be attributed to environmental factors such
as temperature, humidity, or CO\textsubscript{2}, or by some inconsistencies in sample handling. Such factors are likely to cause variations in cell behavior and
inconsistency in the  acquired images. These images were included as-is for SimCLR training, providing an opportunity for the model to learn from realistic conditions. 
However, I didn't artificially introduce additional noise or artifacts through data augmentation. This approach highlights a potential area for further exploration in future work. 

Figure \ref{fig:drug_eff} illustrates that, even with the same drug concentration applied to the spheroid tissue models under controlled conditions, occasional variations in effects were observed across different experiments. These variations could be attributed to slight differences in spheroid composition or undetected microenvironmental influences that affected drug absorption by the spheroid and cellular response. Initiall idea was to evaluate the relative effect of images using the same drug concentration combination as a group, but due to these differences, the ranking assessment needs to be made relative, based on the individual day 10 images, even though the same drug combination was applied.

\begin{figure}[H]
  \centering
  \includegraphics[scale=0.3]{figures/drug_eff.png} 
  \caption{Different morphological changes were observed even when the same specific combination of drugs was applied in two different experiments. The same combination of 40 µM JQ1 and 40 µM GANT61 was applied in two different experiments, named DS61 and DS41}
  \label{fig:drug_eff}
\end{figure}


The table below \ref{tab:dataset} shows the division of different types of image datasets, as explained in Section~\ref{sec:lab-setup}. In this table, Day 8 and Day 9 represent the images obtained during different single-dose and drug-screen experiments. These images are included for SimCLR training. From the table, it is evident that there is a class imbalance.
\begin{table}[H]
  \centering
  \resizebox{\textwidth}{!}{%
  \begin{tabular}{lcccccc}
    \toprule
    \textbf{Class} & \textbf{Drug Screen} & \textbf{Single Dose} & \textbf{Control day 10} & \textbf{Control} & \textbf{Day 8 \& 9} & \textbf{Total} \\ 
    \midrule
    \textbf{No. of Images (\%)} & 200 (16.05\%) & 103 (8.26\%) & 231 (18.54\%) & 472 (37.89\%) & 240 (19.26\%) & 1246 \\
    \bottomrule
  \end{tabular}%
  }
  \caption{Dataset Class Overview}
  \label{tab:dataset}
\end{table}
An 8-bit image encompasses 256 color tones (ranging from 0 to 255) per channel, while a 16-bit image accommodates 65,536 color tones (ranging from 0 to 65,535) per channel, specifically allowing for 65,536 shades of gray. Retaining the original 16-bit depth is crucial because converting it to an 8-bit image for faster and more efficient computation can lead to significant information loss in intensity details. Since 8-bit images permit only 256 possible values, the finer intensity variations present in 16-bit images become compressed, as illustrated in \ref{fig:8bitvs16bit}. For instance, two distinct 16-bit values (ranging from 30,000 to 30,048) could map to the same 8-bit value (for example, both might be mapped to 117). This results in the loss of subtle intensity differences, which can be vital in our image task, where minute variations in intensity can indicate important features, such as the gradual transition of dark color from the center to the border or the amount of debris surrounding the tumor cell.


 \begin{figure}[H]
  \centering
  \includegraphics[scale=0.4]{figures/8bitvs16bit.png} 
  \caption{8-bit vs 16 bit data lose comparison. Although the images visually appear similar, 16-bit data has pixel values ranging from 0 to 65535, while 8-bit data has pixel values ranging from 0 to 255}
  \label{fig:8bitvs16bit}
\end{figure}

\section{Challenges}
\begin{enumerate}
    \item \textbf{Limited Dataset for Day 7 to Day 10 ranking prediction Model:} The dataset for predicting outcomes between Day 7 and Day 10 is limited, which poses a challenge for training and evaluation.

    \item \textbf{Issues with Day 10 Images:}
     Day 10 images can exhibit variations such as flipping, blurring, brightness changes, and position changes from the original position of tumor tissue 
        model in the image. 
        The position change occurs because, when the well plate is brought outside and then placed back under the microscope for taking day 10 image, 
        the position often shifts. This relative position of the tumor cell in the image can shift from the center to other directions.
         While a 'center crop' explained in in Section~\ref{sec:data preprocessing} approach can address this issue to some extent, it fails when the tumor tissue model 
         is located at the edge of the image. To deal this issue for some extend, Applying horizontal and vertical flips, rotations by $90^\circ$ and $270^\circ$, as 
        well as combinations like horizontal flip + rotation $90^\circ$ and horizontal flip + rotation $270^\circ$, can help make the model invariant to position changes.
   

    \item \textbf{Variability in Drug Effects on Day 10:}
   The same drug can have different effects on Day 10. Hence we can't assess the drug efficacy based on the group of images applied by same drug combination instead we need to make the ranking relative 
        assessment based on each individual the day 10 images as illustrated in \ref{fig:drug_eff}.
 
\end{enumerate}

A total of 40 images from the drug screen with significant debris were manually selected for intermediate evaluation of SimCLR, as explained in Chapter \ref{ch:Methodology for Intermediate evaluation of SimCLR model}, and ranking strategies, as explained in Chapter \ref{ch:Methodology for Ranking}.




\chapter{Structure of the Thesis}\label{ch:Structure of the Thesis}

The goal of this thesis is to leverage representation learning of bright-field microscopy images using SimCLR to develop a ranking/ordering scale 
(1 to \( n \)) for all images. 

In this chapter, I will explain the framework designed to address this problem. The task is divided into three main pipelines.

First, we learn latent representations from the images using SimCLR as a self-supervised learning (SSL) model. The details will be explained in the 
\textit{Methodology for SimCLR} chapter.

Next, we perform an intermediate evaluation to assess whether the features have been effectively learned. The intermediate evaluation aims to:

\begin{enumerate}
    \item Classify the images,
    \item Cluster the images, and
    \item Ensure that the same features are extracted for identical images, even after transformations such as flipping, rotation, blurring, or brightness changes,
     using a direct distance measure approach.
\end{enumerate}

The details of this step will be explained in the \textit{Methodology for Intermediate Evaluation} chapter.

Subsequently, we use the learned features to establish a ranking scale. To achieve this, the following methods are employed:

\begin{enumerate}
    \item Prediction model,
    \item K-means centroid approach
    \item Softmax approach, and
    \item PCA variation
\end{enumerate}

The details of these methods will be discussed in the \textit{Methodology for Ranking Strategy} chapter.

Finally, we address an important question: why do we need to learn feature vectors from the images? Could the ranking task perform better using the raw images 
directly instead of SimCLR features? To answer this, we conduct a comparative study to evaluate the performance of both approaches.

\textcolor{red}{Section 6 - it seems to be very focused on the simCLR features. I have thought that you tried similar approaches (e.g. the clustering)
 over some other features, e.g. directly the original images or the ResNet features. Is it not the case? Do you have a baseline model to compare to?}

 So in each intermediate evaluatiions and ranking strategy we will aslo compare the resnet features to original images.

 \textcolor{red}{include original image comparison in the figure}

The overall structure of the thesis is illustrated in Figure~\ref{fig:BigOutline}.

\begin{figure}[H]
  \centering
  \includegraphics[scale=0.46]{figures/bigpic.png} 
  \caption{Overall framework of the thesis.}
  \label{fig:BigOutline}
\end{figure}



 
 
 
 
 
 
 since we don't have ground truth labels to rank the images except control images.



My strategy was to simplify the current ranking problem to
 only scale/order/rank images 
using only control, single dose and exploded images.
The reason to pick these groups is that controls we know that there is no drug applied which means that there is no effect of drug at all. 
single dose images are the category which is clinically recommended at the moment (eventhough we don't know how much drug effected or how 
much it killed the cancer) 
and exploded are visually exploded from the original cancer cell meansing we can visually see debris around the cancer cell potentially which
 may potentially harm the surrounding goof cells.
once if we can order those small subset of entire images, we can add the other image as inference to see where they plotted relative to 
control 
or single dose or exloded in this scale.


\chapter{Methodology for SimCLR}\label{ch: Methodology for SimCLR}

\section{Data preprocessing} \label{sec:data preprocessing}

Detailed study/research/experiments on data augmentation and image preprocessing techniques sepcifically for our 16 bit gray scale image are still need to be done.
Currently, as the focus is on creating the complete pipeline, the standard data augmentation combination (which showed high performance for SimCLR downstream tasks) from the SimCLR \cite{chen2020simple} paper is being used, as shown below.

\begin{enumerate}
  \item  We can't center crop it because for some of cancer cells are not centered in the image instead they are close to egdes. so we have to make 
  sure that when we crop it it should include the cacne cell fully. the solution is find the boundary of cancer cell and get the bounding box of cancer 
  cell then make crop to required size. original image width size is: 2456*2054 (H*W). crop the original image to have H = W. since the cancer cell debris
   spread across the width, we didn't change the width to include the debris,
   instead we reduced the hight to same size of width. hence we get H = W = 2054. if cancer cell is formed for instance, the cell it self spread acroos
    one diemsion that means its not valid cultvation by robot so we can digard it. ie maximum area of cancer cell should be included in this square 
    2054*2054 size image. advantage of croping like this are:
    \begin{enumerate}
      \item  remove unnecessary background which contains no information
      \item  there will be no shear during resize to 96*96 augmentation like in the figure.
      \begin{figure}[H]
        \centering
        \includegraphics[scale=0.46]{figures/long.png} 
        \caption{First row: croped to 2054*2054 then resized to 96*96: no shear change/elongation in one dimension happened, Second row: 
       original image 2456*2054 then resized to 96*96: shear change/elongation in one dimension happened}
        \label{fig:elong}
      \end{figure}
      
      \item rotation = 90 and 270, Horiflip+rotation 90, Horiflip+rotation270 this is only possible because of square image. if its other 
      angles except multiples of 90 then images will have black part for that we need additional careful interpolation or something like that.
       so my point is since the image is square we could take rotations of 90 mulitiples without any additional tasks.
       \begin{figure}[H]
        \centering
        \includegraphics[scale=0.46]{figures/rotation.png} 
        \caption{A: croped to 2054*2054 then resized to 96*96: No black cuts. B: 
        original image 2456*2054 then resized to 96*96: black cuts happened}
        \label{fig:rotation}
      \end{figure}
    \end{enumerate}

  \item Normalize the 16-bit image to [0, 1] for the following reasons:
  \begin{enumerate}
      \item Ideally, normalization should be done at the end after augmentations to ensure scaled input to the neural network, but in our case, we have to normalize first since the augmentation with \texttt{torch.transform.ColorJitter} didn't work without scaled data.
      \item \texttt{torch.transform.ToTensor()} didn't scale the data points to the [0, 1] range.
  \end{enumerate}
  
  \item Perform the following augmentations:
  \begin{enumerate}
      \item Apply a horizontal flip.
      \item Randomly crop the image and resize it to $96 \times 96$. i choosed this small H and W as image size to feed train our model because of two reasons: 1. 
      computational fast, so that we can complete the pipeline in time. 2. if we can get good performance in ranking with this small image sizes (ie less pixel details
       compared to 2054 * 2056 ) that means we can improve the performance with larger image sizes.
      \item Randomly change the brightness, contrast, saturation, and hue of the cropped patch.
  \end{enumerate}

  \item Perform Z-score normalization after data augmentation for the following reasons:
  \begin{enumerate}
      \item Pretrained models require this preprocessing.
      \item It ensures that the data is still normalized even after data augmentation tweaks, allowing for effective feeding into the neural network.
  \end{enumerate}

  \item For each original image, repeat step 2 twice to obtain two augmented images.
\end{enumerate}


Visualisation of before and after preprocessing of image shown in figures 6.5 to 6.11.

\section{Data augmentation} \label{sec:data augmentation}
\textcolor{red}{add what kind of rotations and flips are possible without repetion of image} 
I started with data augmentation just like simclr paper did that is strong data augmentation, just to get 
to play around , beacuse it was easy to do , since I can directly adapt code 
from tutorial just to have complete pipeline.

I found that those strong augmentations transformed to distribution out of 
the original distribution like we see in the figures below. specifically color jitterness including brightness, 
contrast, hue, saturation.

Interestingly, for 16-bit images with 3 channels, instead of a reduction, there was an increase in the number of unique pixel values—by a maximum of 258,757 percentage. 
The issue with this increase is that after data augmentation, the new pixel values are not distributed similarly to the original image. Instead, they shift to
 the two extremes,
 such as 0 or 1, or sometimes pushing values to both 0 and 1, which deviate significantly from the original image distribution, as shown in Figures
  \ref{fig:16bit_three_v1} and \ref{fig:16bit_three_v2}.

  \textbf{16-bit three-channel image before and after data augmentation:}
  \begin{itemize}
    \item Number of unique pixel values in the original image: 2111
    \item Number of unique pixel values in the augmented image: 5044624
    \item Original Image - Minimum pixel value: 0.13064774870872498, Maximum pixel value: 0.6874189376831055
    \item Augmented Image - Minimum pixel value: 0.022128667682409286, Maximum pixel value: 0.11041323840618134
  \end{itemize}
  
  \begin{figure}[H]
    \centering
    \includegraphics[scale=0.5]{figures/16bit_three_1.png} 
    \caption{16-bit three-channel image after 3000 epochs of random color jitter applied using PyTorch.}
    \label{fig:16bit_three_v1}
  \end{figure}
  
  \textbf{Another example of a 16-bit three-channel image before and after data augmentation:}
  
  \begin{itemize}
    \item Original Image - Unique pixel counts per channel: 2137
    \item Augmented Image - Unique pixel counts per channel: 1686717
    \item Original Image - Minimum pixel value: 0.1306, Maximum pixel value: 0.6874
    \item Augmented Image -  Minimum pixel value: 0.1970, Maximum pixel value: 0.3748
  \end{itemize}
  
  \begin{figure}[H]
    \centering
    \includegraphics[scale=0.5]{figures/16bithree2.png} 
    \caption{16-bit three-channel image after 3000 epochs of random color jitter applied using PyTorch.}
    \label{fig:16bit_three_v2}
  \end{figure}
  
  \textbf{16-bit single-channel image before and after data augmentation:}
  \begin{itemize}
    \item Number of unique pixel values in the original image: 2111
    \item Number of unique pixel values in the augmented image: 1058
    \item Original Image - Minimum pixel value:  0.13064774870872498, Maximum pixel value: 0.6666666865348816
    \item Augmented Image - Minimum pixel value: 0, Maximum pixel value: 1
  \end{itemize}
  
  \begin{figure}[H]
    \centering
    \includegraphics[scale=0.5]{figures/16bit_onen.png} 
    \caption{8-bit single-channel image after 3000 epochs of random color jitter applied using PyTorch. Reduction percentage in unique pixel values: 49.88\%}
    \label{fig:16bit_single_channel}
  \end{figure}

  $49\%$ maximum reduction for 16-bit single-channel data augmentation with color jitter is also not ideal, as it diminishes the gradual spread of 
  darker regions 
as happened in original image, as  as observed in figure \ref{fig:16bit_single_channel}. One potential solution is to experiment with specific parameters
 within the color 
jitter transform instead of using random values, ensuring that the reduction in the number of unique pixel values does not exceed, for example, $30\%$. 
Another 
option would be to write a custom Python function, depending on the available time. Other augmentations from PyTorch work fine in this experiment.

then I removed contrast, saturation and hue. because if I tried to make 
those augs as invariant that means I'm making control and treated as 
similar because if I increase or decrease contrast or 
saturation or hue it increase/decrease the darkness of cancer cell to 
have similarity.
which doesn't make sense because we want exact opposite of it.

but if I use the same croping percentage as the original simclr does, 
that means I crop 0.1  to 1 percentage which doesn't make sense because that 
menas we make smilarity to small background to cancer which is unnecessary. 
so i chaneg the percentage to 0.4-1.

Questionable below one:

lasty it may be possible to learn efficient feature extraction by not 
applying cropping, ie learn by seeing full picture (big picture). 
I understand this is not good argument because then dog and cat have same 
color and maybe similar shape but still learns to differentiate using simclr.
But maybe for time predictoin ranking model it helps. because model have to 
predict the change from day 7 to day 10. and most of the ime the size/shape
 changes. especially to the category which are exploded, they produce debris. 
 so maybe data augmentation seeing the time change. 





\section{Train SimCLR as SSL model}
For step 1, as explained in chapter \ref{ch: Methodology}, SimCLR was used as the first model for self-supervised learning (SSL). 
Future work: Later, other models such as masked autoencoders and DINO will be explored, depending on the available time.
Why we would like to try other models? Because SimCLR demands larger batch size and more data for better performance which we don't have.



\subsubsection{Model}
The Resnet18 \cite{he2015deepresiduallearningimage} model processes a single image to produce a latent representation of the input, aiming to cluster 
similar images together in a latent space. 

\subsubsection{Training }
The training process follows these steps:

\begin{enumerate}
    \item We take a batch of images with batch size $N$.
    
    \item Our dataset class returns two augmented versions for each original image as explained in section \ref{sec:data preprocessing} in the batch, 
    resulting in $2N$ images as input.

    \item The model produces $2N$ latent representations, independently for each augmented image.

    \item For each batch, the two augmentations of the same image are treated as positive pairs, while all others are considered negative pairs.

    \item We calculate the cosine similarities between the positive and negative pairs. These cosine similarities are then used as input to the loss
     function described below equation \ref{eq:loss}
\end{enumerate}
The original loss function for each pair from SimCLR paper \cite{chen2020simple} is defined as:

\begin{equation}
\ell_{i, j} = -\log \frac{\exp \left(\operatorname{sim}\left(\boldsymbol{z}_i, \boldsymbol{z}_j\right) / \tau\right)}{\sum_{k=1}^{2 N} \mathbf{1}_{[k \neq i]} \exp \left(\operatorname{sim}\left(\boldsymbol{z}_i, \boldsymbol{z}_k\right) / \tau\right)}
\label{eq:original}
\end{equation}
  
which we can reformualte as:

1. Apply the logarithm: The negative log of a fraction can be separated into the difference of the logarithms:
\[
\ell_{i, j} = -\left( \log \left(\exp \left(\operatorname{sim}\left(\boldsymbol{z}_i, \boldsymbol{z}_j\right) / \tau\right)\right) - \log\left( \sum_{k=1}^{2 N} \mathbf{1}_{[k \neq i]} \exp \left(\operatorname{sim}\left(\boldsymbol{z}_i, \boldsymbol{z}_k\right) / \tau\right) \right) \right)
\]

2. Simplifying the first term: The logarithm of an exponential function simplifies as follows:
\[
-\log\left(\exp\left(\operatorname{sim}\left(\boldsymbol{z}_i, \boldsymbol{z}_j\right) / \tau\right)\right) = -\frac{\operatorname{sim}\left(z_{i}, z_{j}\right)}{\tau}
\]

Substituting that back into the equation:
\[
\ell_{i, j} = -\frac{\operatorname{sim}\left(z_{i}, z_{j}\right)}{\tau} - \log\left(\sum_{k=1}^{2 N} \mathbf{1}_{[k \neq i]} \exp\left(\operatorname{sim}\left(z_{i}, z_{k}\right) / \tau\right)\right)
\]

This gives us:

\begin{equation}
\ell_{i, j} = -\frac{\operatorname{sim}\left(z_{i}, z_{j}\right)}{\tau} + \log\left[\sum_{k=1}^{2 N} \mathbf{1}_{[k \neq i]} \exp\left(\operatorname{sim}\left(z_{i}, z_{k}\right) / \tau\right)\right]
\label{eq:loss}
\end{equation}

where \(\mathbf{1}_{[k \neq i]} \in \{0, 1\}\) is an indicator function evaluating to 1 iff \(k \neq i\), and \(\tau\) denotes a temperature parameter. The final loss is computed across all positive pairs, both \((i, j)\) and \((j, i)\), in a mini-batch with $z_i, z_k$ representing negative pairs.

Equation \ref{eq:loss} implemented as the loss function in our experiments.

The above standard SimCLR loss function and data augmentation combination  with the ResNet18 model will be used as the initial benchmark for experiments, and in the future, the following variations will be explored.

\textbf{Variation ideas:}

\begin{enumerate}
  \item Explore different data augmentation combinations by researching the best augmentations for medical grayscale images, and applying intuitive approaches beyond the standard SimCLR \cite{chen2020simple} data augmentation combinations as explained in section \ref{sec:data preprocessing}.
  \item Each image is treated as an RGB image with 3 channels, and two of the best-performing data augmentations, which yielded high performance for our downstream task.

  \item One channel is considered as the anchor (the most sharpened layer), and the others are treated as the two augmentations.
  \item One channel is considered as the anchor (the most sharpened layer), and two of the best-performing data augmentations, which yielded high performance for the our downstream task.
  \item Supervised SimCLR: Ensure that no images from the same breed/class are included in the negative samples.
  \item Since SimCLR architecture \cite{chen2020simple} allows for flexibility in model selection, and explore other pretrained models  than Resnet18  suitable for medical grayscale images. For example pretrained U-Net \cite{ronneberger2015unetconvolutionalnetworksbiomedical} model for MRI brain images from PyTorch.  
  \item Include the anchor as a positive sample, i.e., 3 augmentations in total (1 anchor as augmentation and the other 2 layers as augmentations). This resembles to triplet loss (not sure, need to be studied)  
\end{enumerate}
For variations 5 and 7 we need to modify the loss function since it includes more than 2 augmentations.

\textbf{Variations implementations:}  \label{sec:variations_implementations}

The two variations tried so far differ only in how they handle the image for data augmentation. 

In the first variation, we take a 3-channel image and treat it like a standard RGB image, applying SimCLR-style augmentations to create two augmented versions.

In the second variation, we take a 3-channel image and compute the sharpness of each layer by calculating the magnitude of the gradient of pixel intensities in the x and y directions, which indicates edge strength and provides a measure of how sharp the transitions between pixel values are. The sharpest layer is used as the anchor, while the other two layers are treated as augmentations. 

\subsubsection{Variation 1:}
\textbf{Input to model (train loader dimension) :} 

\begin{itemize}
  \item aug1: torch.Size([16, 3, 96, 96])        (batch size, no of channels, H, W)
  \item aug2: torch.Size([16, 3, 96, 96])        (batch size, no of channels, H, W) \vspace{1em}
\end{itemize} \vspace{1em}
\textbf{Model output just after convolution layers: (before applying projetion head)} 
\begin{itemize}
  \item torch.Size([16, 512, 1, 1]) (Batch size, standard resnet18 output dimension after avg pooling, H,W)   
  \item This output feature will be used for further downstream task.  \vspace{1em}
\end{itemize}

\textbf{Model output after projection head:}
\begin{itemize}
  \item torch.Size([16, 20])  (Batch size, no of values in feature vector)  
  \item No of values in feature vector is a variable which we can change and experiment which will give better accuracy.
\end{itemize}

The figure below shows the anchor and its two augmented versions as explained in section \ref{sec:data preprocessing}.

\begin{figure}[H]
  \centering
  \includegraphics[width=0.9\linewidth]{figures/3_1.png} % Adjust the width as needed
  \caption{Sample 1: Anchor (the preprocessed original image) with 3 channels and its augmentations}
  \label{fig:augmentation}
\end{figure}


  \begin{figure}[H]
    \centering
    \includegraphics[width=0.9\linewidth]{figures/3_2fine.png} % Adjust the width as needed
    \caption{Sample 2: Anchor (the preprocessed original image) with 3 channels and its augmentations}
    \label{fig:augmentations}
  \end{figure}
\subsubsection{Variation 2:}

\textbf{Input to model (train loader dimensions) :} 
\begin{itemize}
   \item aug1: torch.Size([16, 1, 96, 96])        (batch size, no of channels, H, W)
   \item aug2: torch.Size([16, 1, 96, 96])        (batch size, no of channels, H, W) \vspace{1em}
\end{itemize}
\textbf{Model output just after convolution layers: (before applying projetion head)} 
\begin{itemize}
  \item torch.Size([16, 512, 1, 1]) (Batch size, standard resnet18 output dimension after avg pooling, H, W)   
  \item This output feature will be used for further downstream task.  \vspace{1em}
\end{itemize}

\textbf{Model output after projection head:}
\begin{itemize}
  \item torch.Size([16, 20])  (Batch size, no of values in feature vector)  
  \item No of values in feature vector is a variable which we can change and experiment which will give better accuracy.
\end{itemize}

The figure below shows the anchor and its two augmented versions as explained in section \ref{sec:data preprocessing}.
\begin{figure}[H]
  \centering
  \includegraphics[width=0.9\linewidth]{figures/1_1.png} % Adjust the width as needed
  \caption{Sample 1: Anchor (the preprocessed sharpest layer amoung all 3 layers) with one channel and its augmentations}
  \label{fig:1doutput1}
\end{figure}

\begin{figure}[H]
  \centering
  \includegraphics[width=0.9\linewidth]{figures/1_2.png} % Adjust the width as needed
  \caption{Sample 2: Anchor (the preprocessed sharpest layer amoung all 3 layers) with one channel and its augmentations}
  \label{fig:1doutput3}
\end{figure}





\chapter{Methodology for Intermediate evaluation of SimCLR model}\label{ch:Methodology for Intermediate evaluation of SimCLR model}

\textcolor{red}{change the name to evaluatoin to check whether simclr learned something? instead of intermediate evaluatoin?}  

If we do kmeans with cosine distance distance then it only compare direction of the vector thats why we need to do normal kmeans with euclidean dist to show 
the magnitude similarity

Final evaluation of the SimCLR model depends on the  Ranking task, neverthless we can use other evaluation metrics, such as downstream task 
like classification  and clusetring to check if simclr atleast learned to differentiate between untreated , single dose and exploded images.

Distance measurement approach is used to whats the performance of simclr feats derived from  5 different aug pipelines  in terms of cosine distance between 2 augs derived 
from same image.

\section{Classification using Logistic Regression on the SimCLR features}
1. A common approach to verify whether the SSL model has learned generalized representations is to perform Logistic Regression on the learned features.
 In other words, we use a single, linear layer that maps these representations to class predictions, where the three categories, 'untreated' and 'single dose', 'explod'  
 serve as our classes. The Logistic Regression model can only perform well if the learned representations capture all the relevant features necessary for the task. 
 Moreover, we don't need to worry much about overfitting since only a few parameters are trained. Therefore, we expect the model to perform well even with limited data.
 

2. Baseline comparison to original images. We classify the original images to see if simclr feature vectors are better or worse than original image to classify.

The logistic regression model is implemented as a single linear layer, where the input dimension corresponds to the feature dimension of feature spits out from simclr, and the 
output dimension corresponds to the number of classes. Specifically, the model uses a feature dimension of \(512\) if the feature is before projection head or  \(20\) if the
 feature is after projection head and outputs predictions for \(3\) classes. The mathematical representation of the model is as follows:

\[
\hat{y} = xW^T + b
\]

where:

- \(x \in \mathbb{R}^{N \times d}\) is the input feature vector, where \(d = 512\) if the feature is extracted before the projection head, or \(d = 20\) if the feature is 
extracted after the projection head.  
- \(W \in \mathbb{R}^{3 \times d}\) represents the learnable weights.  
- \(b \in \mathbb{R}^3\) is the bias term,
- \(\hat{y} \in \mathbb{R}^{N \times 3}\) are the logits representing the unnormalized class scores for \(N\) samples.

PyTorch's CrossEntropyLoss combines the softmax operation with the log loss computation for numerical stability, which is why we pass the logits (unnormalized outputs) directly
to the loss function.

This model is trained to minimize the cross-entropy loss for multi-class classification.
3 classes trained for 250 epochs. The dataset was divided into a training set (80\%) and a validation set (20\%). Batch size = 8. loss function = 
cross entropy loss.  a learning rate of \( 5 \times 10^{-4} \). 

The learning rate scheduler used is the \texttt{MultiStepLR}, which reduces the learning rate at specific milestones during training. In this case, the learning rate is 
reduced by a factor of 0.1 at the epochs corresponding to 60\% and 80\% of the total training epochs. These milestones are defined as:
\[
\text{milestones} = \left[ \text{int}(T_{\text{max}} \times 0.6), \, \text{int}(T_{\text{max}} \times 0.8) \right]
\]
where \( T_{\text{max}} \) represents the total number of training epochs. The \texttt{gamma} parameter specifies the factor by which the learning rate is multiplied at each 
milestone, which in this case is 0.1.

\begin{table}[h!]
    \centering
    \caption{Dataset Summary from Drug Screening Experiments}
    \label{tab:dataset_summary}
    \begin{tabular}{lc}
    \toprule
    \textbf{Dataset}                              & \textbf{Total Number of Images} \\ 
    \midrule
    Untreated Dataset                             & 472                            \\ 
    Single Dose Dataset                           & 103                            \\ 
    Exploded Dataset from Drug Screening Experiments & 40                             \\ 
    \bottomrule
    \end{tabular}
\end{table}

I didn't add any other comibnation of drug screening experiments since we are not sure whether they belong to which group. Some of them have medium resemblence of the
 above 3 classes but we can't be sure.


\subsection{Comparison of Classification Accuracy and Epochs for Different Data Augmentations}

As explained in Data augmentation section we use  'strong', 'sweet', 'resize', 'sweet no contrast', 'resize no contrast' simclr features to compare their performance inbetween
 and also against original images which one classify these 3 classes better.

We calculate the train accuracy and the test accuracy for each data aug pipelines as well as for original images.

Train epoch in the table: The number of epochs required to reach the best training accuracy.
Test epoch in the table: The number of epochs required to reach the best test accuracy.

\textcolor{red}{how do we turn raw images to orig images to feats}
To use original images directly first we resized to 96*96 since we did that for simclr so that it could be fair comparison. Then we normalised each by dividing it by 65535 
(16 bit). Each image flatten into vectors of ([1,96*96*3]where 96 =H=W and 3 = no of channels) suitable for logistic regression model. and we use same parameters that we used 
for simclr feats classification lke learning rate , batch size etc for fair comparison.


\textcolor{red}{Table 8.1.1 - are you saying that you classification accuracy is 100?
 That seems far too good to be true. Why do the train runs have different number of epochs? And what do you mean by "Test Epoch"?!}


 \textcolor{red}{general comment to tables with result - after giving the tables you need to write a text which helps us to interpret it. What are the most important 
 number we shall look at? What shall we take out from the table as a message. For example something like (please do not use this directly. I have no clue 
 if this is the message you want us to take away from the table) "In Table 8.1. we show that when using the features Before Projection Head all 
 augmentation methods reached 100 train and test accuracy. The Resize approach needed the fewest epochs."}
    
    \begin{table}[h!]
        \centering
        \resizebox{\textwidth}{!}{%
        \begin{tabular}{llcccccc}
        \toprule
        \textbf{Augmentation Type}      & \textbf{Metric} & \textbf{Strong} & \textbf{Sweet} & \textbf{Resize} & \textbf{Resize No Contrast} & \textbf{Sweet No Contrast} \\ \midrule
        \multirow{4}{*}{\textbf{After Projection Head}}  
            & Train Accuracy (\%) & 66.67 & 47.76 & 89.63 & 63.41 & 54.67 \\
            & Train Epoch         & 246   & 249   & 250   & 250   & 245   \\
            & Test Accuracy (\%)  & 68.29 & 55.28 & 90.24 & 63.41 & 56.10 \\
            & Validation Epoch          & 228   & 246   & 223   & 202   & 244   \\ \midrule
        \multirow{4}{*}{\textbf{Before Projection Head}} 
            & Train Accuracy (\%) & 100 & 100 & 100 & 100 & 100 \\
            & Train Epoch         & 190    & 3     & 13     & 12     & 21     \\
            & Validation Accuracy (\%)  & 100 & 100 & 100 & 100 & 100 \\
            & Validation Epoch          & 1      & 1      & 1      & 2      & 2      \\ 
        \bottomrule
        \end{tabular}%
        }
        \caption{Performance metrics for different augmentation strategies before and after the projection head.}
        \label{tab:augmentation_metric}
        \end{table}
        

        \begin{table}[h!]
            \centering
            \caption{Original Image Results}
            \label{tab:original_image_results}
            \begin{tabular}{lcccc}
            \toprule
            \textbf{Metric}         & \textbf{Train Accuracy (\%)} & \textbf{Train Epoch} & \textbf{Validation Accuracy (\%)} & \textbf{Validation Epoch} \\ \midrule
            \textbf{Original Image} & 99.39                        & 249                  & 99.19                        & 17                  \\ 
            \bottomrule
            \end{tabular}
        \end{table}
           
Train loss vs Val loss . Train accurcay  vs val accuracy 
          

          \begin{figure}[H]
            \centering
            \includegraphics[scale=0.95]{figures/before.pdf} 
            \caption{before}
            \label{fig:before}
          \end{figure}

          \begin{figure}[H]
            \centering
            \includegraphics[scale=0.95]{figures/after.pdf} 
            \caption{after}
            \label{fig:after}
          \end{figure}

          
          \begin{figure}[H]
            \centering
            \includegraphics[scale=0.425]{figures/orig_classi.png} 
            \caption{orig}
            \label{fig:orig}
          \end{figure}

          \begin{figure}[H]
            \centering
            \includegraphics[scale=0.5]{figures/test_class.png} 
            \caption{before}
            \label{fig:before}
          \end{figure}

          \begin{figure}[H]
            \centering
            \includegraphics[scale=0.5]{figures/train_class.png} 
            \caption{before}
            \label{fig:before}
          \end{figure}
\subsection{Inference}

I choose my self 22 images that have close resemblence in my eyes to single dose images from drug screen for inference. ( I repeat these are not biology expert labeled drug 
screened image. so any inference result is irrelavant in the sense to actual ground truth )

For doing inferecne I choosed before head projection features since thats the one gave 100 perentage accuracy while train and validation.


\begin{table}[H]
    \centering
    \begin{tabular}{@{}lcccccc@{}}
    \toprule
    \textbf{Type} & \textbf{Strong} & \textbf{Sweet} & \textbf{Resize} & \textbf{Resize no Contrast} & \textbf{Sweet no Contrast} \\ \midrule
    Before Projection Head & 90.91 & 100.00 & 100.00 & 100.00 & 95.45 \\ \midrule
    Original Images & & & & 68.18 & \\ \bottomrule
    \end{tabular}
    \caption{Performance Metrics Before and After the Projection Head}
    \label{tab:performance_metrics}
\end{table}



From the above table we can see that during the inference except 'strong' and 'sweet no contrast' all other augs still able to reach 100 percentage accuracy to classify the ds 
close to sd images to sd based on visual resemblence. Using raw Original images features it got  68.18 percentage accuracy while Simclr before projectoin head could classify atleast 90 percentage.


\textcolor{red}{table 8.2 can you somehow bring this into table 8.1 e.g. as another column? Would be easier to read. The first column 
"Augmentation Type" could be made smaller if you wrap the text so that it is on multiple lines.} 


In Table 8.1. we show that when using the features Before Projection Head, all augmentation methods reached 100 percent train and test accuracy. It is clear that the features extracted 
before the projection head perform far better than those extracted after the projection head, as seen in the original similar paper. Also, the number of epochs required
to reach the best accuracy is lower for the features extracted before the projection head than those extracted after the projection head.

While original images also classify with 99 percent accuracy, they perform better than the features we use after projection. Also, their almost indistinguishable performance 
from the features extracted before the projection head leads to the conclusion that we cannot make an informed decision based on this classification task.



\section{kmeans clustering}

Idea is whether the learned representation from simlcr outperforms the original images in clustering the images (unsupervised manner) For that we use simple kmeans 
clusetring. We will cluster them based on both euclidean distance as well as cosine distance.





\subsection{Evaluation}

Full dataset (unbalanced): contains all control(untreated): 472, all single dose: 103, all exploded: 40 images, all drug screen visualy similar/closer to single dose.

40 subset (Balanced to the minimum nof set in the group which is exploded. ie there is only total 40 of exploded): contains random but make sure it have exploded like 
controls total 40, and all exploded which is basically 40, 10 ds close to sd, 30 single dose.

Curated Full dataset (unbalanced): contains all control except controls have debris: 280, all single dose: 103, all exploded: 40 images, all drug screen visualy 
similar/closer to single dose: 22.

Curated 40 subset  (Balanced to the minimum nof set in the group which is exploded. ie there is only total 40 of exploded but excluded explod look alike from control):
contains random but make sure it doesn't have exploded like controls total 40, and all exploded which is basically 40, 10 ds close to sd, 30 single dose.

\begin{table}[H]
\centering
\caption{Summary of Datasets}
\label{tab:dataset_summary}
\resizebox{\textwidth}{!}{%
\begin{tabular}{lcccc}
\toprule
\textbf{Dataset}             & \textbf{Control (C)} & \textbf{Single Dose (SD)} & \textbf{Exploded (E)} & \textbf{DS Closer to SD (DS-SD)} \\ \midrule
Full Dataset (Unbalanced)    & 472                  & 103                       & 40                    & Included                          \\ 
40 Subset (Balanced)         & 40                   & 30                        & 40                    & 10                                \\ 
Curated Full Dataset         & 280                  & 103                       & 40                    & 22                                \\ 
Curated 40 Subset (Balanced) & 40                   & 30                        & 40                    & 10                                \\ 
\bottomrule
\end{tabular}%
}
\end{table}

    

run 100 times for different random initilisation.

\begin{table}[htbp]
    \centering
    \caption{Dataset Composition Overview}
    \begin{tabular}{lccccl}
    \toprule
    Dataset Type & Control & Single Dose & Exploded & Drug Screen & Notes \\
    \midrule
    \multicolumn{6}{l}{\textbf{Raw Datasets}} \\
    \midrule
    Full & 472 & 103 & 40 & -- & Unbalanced \\
    Balanced-40 & 40 & 30 & 40 & 10 & Minimum set balanced \\
    \midrule
    \multicolumn{6}{l}{\textbf{Curated Datasets}} \\
    \midrule
    Full & 280 & 103 & 40 & 22 & Debris-free controls \\
    Balanced-40 & 40 & 30 & 40 & 10 & No exploded-like controls \\
    \bottomrule
    \end{tabular}
    \begin{flushleft}
    \small
    Note: Drug screen images are visually similar to single dose treatment. All balanced datasets are normalized to the exploded group size (n=40).
    \end{flushleft}
    \end{table}

    \begin{table}[H]
        \centering
        \caption{Evaluation Results on Different Datasets and Augmentations with cosine distance}
        \label{tab:evaluation_results_cosine_distance}
        \resizebox{\textwidth}{!}{%
        \begin{tabular}{@{}llcccc@{}}
            \toprule
            \textbf{Projection Head} & \textbf{Augmentation Type} & \textbf{Full Dataset (Unbalanced)} & \textbf{40 Subset (Balanced)} & \textbf{Curated Full Dataset} & \textbf{Curated 40 Subset} \\ \midrule
            \multirow{5}{*}{\textbf{Before}} 
            & Strong                & 99.18 & 100 & 100 & 100 \\
            & Sweet                 & 99.18 & 100 & 100 & 100 \\
            & Resize                & 65.7 & 97.5 & 97.87 & 100 \\
            & Resize No Contrast    & 69.92 & 90 & 86.76 & 100 \\
            & Sweet No Contrast     & 96.74 & 99.17 & 99.52  & 99.17 \\ \midrule
            \multirow{5}{*}{\textbf{After}} 
            & Strong                & 60.04 & 92.5 & 55.08 & 100 \\
            & Sweet                 & 48.13 & 67.5 & 53.65 & 71.67 \\
            & Resize                & 59.83 & 92.50 & 74.23 & 100 \\
            & Resize No Contrast    & 45.85 & 58.33 & 45.62 & 62.5 \\
            & Sweet No Contrast     & 49.26 & 82.5 & 57.21 & 85.83 \\ \bottomrule
        \end{tabular}%
        }
    \end{table}
    
    \begin{table}[H]
        \centering
        \caption{Evaluation Results on Different Datasets and Augmentations with euclidean distance}
        \label{tab:evaluation_results_euclidean}
        \resizebox{\textwidth}{!}{%
        \begin{tabular}{@{}llcccc@{}}
            \toprule
            \textbf{Projection Head} & \textbf{Augmentation Type} & \textbf{Full Dataset (Unbalanced)} & \textbf{40 Subset (Balanced)} & \textbf{Curated Full Dataset} & \textbf{Curated 40 Subset} \\ \midrule
            \multirow{5}{*}{\textbf{Before}} 
            & Strong                & 83.74 & 100 & 100 & 100 \\
            & Sweet                 & 98.05 & 91.67 & 99.29 & 95.83 \\
            & Resize                & 70.24 & 79.17 & 66.66 & 87.50 \\
            & Resize No Contrast    & 59.51 & 90.00 & 86.28 & 100 \\
            & Sweet No Contrast     & 99.02 & 97.50 & 99.76 & 99.17 \\ \midrule
            \multirow{5}{*}{\textbf{After}} 
            & Strong                & 71.70 & 100 & 72.57 & 100 \\
            & Sweet                 & 47.00 & 68.33 & 52.71 & 74.17 \\
            & Resize                & 43.25 & 83.33 & 53.90 & 87.5 \\
            & Resize No Contrast    & 48.45 & 59.17 & 44.20 & 60.83 \\
            & Sweet No Contrast     & 49.75 & 81.67 & 52.00 & 81.67 \\ \bottomrule
        \end{tabular}%
        }
    \end{table}
    
    \begin{table}[H]
        \centering
        \caption{Evaluation Results Using Different Distance Metrics}
        \label{tab:distance_metrics}
        \resizebox{\textwidth}{!}{%
        \begin{tabular}{@{}llcccc@{}}
            \toprule
            & \textbf{Metric} & \textbf{Full Dataset (Unbalanced)} & \textbf{40 Subset (Balanced)} & \textbf{Curated Full Dataset (Unbalanced)} & \textbf{Curated 40 Subset (Balanced)} \\ \midrule
            \multirow{2}{*}{\textbf{Original Images}} 
            & Cosine Distance    & 60.16 & 69.17 & 58.15 & 70.00 \\ 
            & Euclidean Distance & 55.28 & 72.5 & 53.65 & 75.00 \\ \bottomrule
        \end{tabular}%
        }
    \end{table}
    



\textcolor{red}{Include PCA 1 here? maybe in ranking. but we can add PCA 2 fugures}

\textcolor{red}{do sd VS others all?}

Inference

\begin{table}[H]
    \centering
    \caption{Evaluation Results on COSINE}
    \label{tab:professional_table}
    \resizebox{\textwidth}{!}{%
    \begin{tabular}{@{}llcccc@{}}
        \toprule
        \textbf{Type} & \textbf{Augmentation} & \textbf{Full Dataset (Unbalanced)} & \textbf{Uncured Balanced} & \textbf{Curated Full Dataset} & \textbf{Curated Balanced} \\ 
        \midrule
        \multirow{5}{*}{\textbf{Before Projection Head}} 
        & Strong             & 92.62 & 100 & 100 & 100 \\ 
        & Sweet              & 98.74 92.00 & 100 & 100 & 100 \\ 
        & Resize             & 67.03 & 97.50 & 98.42 & 100 \\ 
        & Resize No Contrast &  71.89    & 90.83    & 87.42     & 100 \\ 
        & Sweet No Contrast  & 97.33     & 98.33    & 99.55     & 99.17   \\ 
        \midrule
        \multirow{2}{*}{\textbf{After Projection Head}} 
        & Strong             & -     & -    & -     & 94.17 \\ 
        & Resize             & -     & -    & -     & 100    \\ 
        \bottomrule
    \end{tabular}%
    }
\end{table}

\textcolor{red}{How do you know in inference dsclose was the one confused? maybe confusion matrix?  sweet no contrast before cosine}

\begin{table}[H]
    \centering
    \caption{Evaluation Results on Euclidean}
    \label{tab:professional_table}
    \resizebox{\textwidth}{!}{%
    \begin{tabular}{@{}llcccc@{}}
        \toprule
        \textbf{Type} & \textbf{Augmentation} & \textbf{Full Dataset (Unbalanced)} & \textbf{Uncured Balanced} & \textbf{Curated Full Dataset} & \textbf{Curated Balanced} \\ 
        \midrule
        \multirow{5}{*}{\textbf{Before Projection Head}} 
        & Strong             & 80.69 & 100 & 99.10 & 100 \\ 
        & Sweet              & 98.12 & 91.67 & 99.32 & 95.83 \\ 
        & Resize             & 77.08 & 79.17 & 70.78 & 86.67 \\ 
        & Resize No Contrast & 68.28 & 100 & 83.59 & 100 \\ 
        & Sweet No Contrast  & 99.06 & 97.50 & 9.78 & 99.17 \\ 
        \midrule
        \multirow{2}{*}{\textbf{After Projection Head}} 
        & Strong             & - & 92.5 & - & 95 \\  
        \bottomrule
    \end{tabular}% 
    }
\end{table}



\section{Direct day 7 to day 10 distance evaluation}\label{sec: distance}
calculate distance between day 7 untreated and its corresponding day 10 treated one.
Idea is:
1. it should give same distance between day 7 and its corresponding day 10
 even its changed in position/flipped. ie checking whether simlcr learned to 
 be invariant to position error in microscope error because of manual 
 handling/transporting.
2. it sould give same distance even if its blured/sharpened
3. it should give sma edistance even if its changed in brightness
4. main test for k means centroid approach. it should give different distance to single dose and exploded.


basic evaluations for ranking:
1.
whether it learned to be invariant the position change:
do flipps and calculate the cosine distance from control to treated flipped versions. ( i don't expect it learn to the change in center of position, 
if it learns good we can say center crop have some effect maybe? but not for the edge one?)
2. 
shape invariant:control to all single dose should be almost same cosine distance.

it also applicable to time prediction and reconstruction ranking evaluation and also from kmeans centriod approach.


\textcolor{red}{We will start with cosine distance and euclidean distance. other distcances based on time if we have}


\textcolor{red}{we are trying to find whether simclr learns something, from simclr side cond 7 to same cond7 augs should have high cosine similarity and within all
 gps foe example within sd higher cosine similarity since its gping this is what we have to check but we will do this only after ranking strategy depend on time}


 \textcolor{red}{then if have time we will do cond7 to cond10, sd stuuf}

 \begin{table}
 \centering
 \begin{tabular}{ll*{5}{l}}
 \toprule
 \multicolumn{2}{l}{Augmentation type} & Strong & Sweet & Resize & No contrast Resize & No contrast sweet \\ 
 \midrule
 & Flip and rotation & 0.1932 & 0.0109 & 0.134 & 0.0120 & 0.0108 \\
 Before & Blur and sharpness & 0.0061 & 0.0185 & 0.171 & 0.0024 & 0.0058 \\
 & Brightness change & 0.0003 & 0.0037 & 0.184 & 0.0013 & 0.0043 \\
 \midrule
 & Flip and rotation & 0.3435 & 0.0204 & 0.0747 & 0.0113 & 0.0212 \\
 After & Blur and sharpness & 0.0108 & 0.0388 & 0.2087 & 0.0047 & 0.0133 \\
 & Brightness change & 0.0005 & 0.0067 & 0.2864 & 0.0024 & 0.0078 \\
 \bottomrule
 \end{tabular}
 \end{table}
 ```
    

 \begin{table}[H]
    \centering
    \begin{tabular}{@{}lll@{}}
    \toprule
    \textbf{Flip and Rotation} & \textbf{Blur and Sharpness} & \textbf{Brightness Change} \\ \midrule
    0.0291                     & 0.0001                     & $5.73 \times 10^{-8}$      \\ \bottomrule
    \end{tabular}
    \caption{Metrics for different augmentation types.}
    \label{tab:augmentation_metrics}
\end{table}
    



  
\chapter{Methodology for Ranking}\label{ch:Methodology for Ranking}
 
since we don't have ground truth labels to rank the images except control images.
My strategy was to simplify the current ranking problem to only scale/order/rank images 
using only control, single dose and exploded images.
The reason to pick these groups is that controls we know that there is no drug applied which means that there is no effect of drug at all. 
single dose images are the category which is clinically recommended at the moment (eventhough we don't know how much drug effected or how 
much it killed the cancer) 
and exploded are visually exploded from the original cancer cell meansing we can visually see debris around the cancer cell potentially which
 may potentially harm the surrounding goof cells.
once if we can order those small subset of entire images, we can add the other image as inference to see where they plotted relative to 
control 
or single dose or exloded in this scale.

\section{Ranking strategy 1: Using CAE}

\subsection{Day7 to day7 reconstruction}
Original images: Unfortunately, the inference loss/metric will ve same beacuse control day 10 and treatd looks same . I was dumb enough to do that.
 thats why i need  day 7 to day 7 reconstruction model.



I start with classical anomaly detection approach that we train a model to 
reconstruct/predict day 10 image from day 10 image exclusively on the untreated images. Since the day 10 prediction model is trained solely on untreated images,
 we expect the the inference loss/metric (i.e., the difference between the predicted and actual Day 10 image) will be very small for untreated images.
Conversely, the inference loss/metric will increase for treated images as their predictions deviate from those of untreated images.
 This inference loss/metric will be used as the feature for the ranking/order scale, where the initial images will start 
 with untreated images that have very small inference loss/metric, and the scale will end with images having high inference loss/metric in ascending order.


\subsection{Day7 to day10 predcition} \label{subsec:day7-to-day10}
\subsubsection{day10 predcition}
\begin{enumerate}
    \item \textbf{Step 1:} Create a latent space representation of all images, including untreated, clinically recommended, 
    and drug screening images, using SimCLR. 
    The idea is that SimCLR effectively learns efficient features of similar images that are not captured by 
    human-interpretable metrics. We expect the SimCLR feature vectors of similar images will be closer in the latent space. 
    In other words, feature vectors of similar images will be more linearly separable.
  
  \item \textbf{Step 2:} Train a prediction model exclusively on the representations of 
  untreated images from Day 7 to Day 10. ( Input: Day 7 feature vector and target: Day 10 feature vector )

  
  \item \textbf{Step 3:} Perform inference on the representations of test images, which include untreated, clinically recommended, and drug screening images.
  \item \textbf{Step 4:} Perform step 2 and step 3 on images instead of simclr feature vectors for comparitive study.
\end{enumerate}

Since the day 10 prediction model is trained solely on the representations of untreated images, the inference loss/metric 
(i.e., the difference between the predicted and actual Day 10 image representations) will be very small for untreated images.
 Conversely, the inference loss/metric will increase for treated images as their representations deviate from those of untreated images.
This inference loss/metric will be used as the feature for the ranking/order scale, where the initial images will start 
with untreated images that have very small inference loss/metric, and the scale will end with images having high inference loss/metric in ascending order. 
Determining a reasonable inference loss/metric will be one of the research problems to tackle.


\section{Ranking strategy 2: K means centroid approach}

This strategy utilizes the after projection head vectors since simclr loss function designed that after projection head vectors have cosine similarity between similar 
group.

\begin{enumerate}
  \item \textbf{Step 1:} Feed control images into k means and find the centriod of control (untreated) cluster based on both cosine distance and the euclidean distance. 
  ( we can choose the distacne metric if we have time)
  
  \item \textbf{Step 2:} calculate the euclidean/cosine distance from this centroid to every images.
  
  \item \textbf{Step 3:} Perform the same operation for simlcr features and on original images.
\end{enumerate}


\subsection*{Group-Wise Ranking Accuracy: Mathematical Definition and Process}

\textcolor{red}{below maybe wrong because I didN't added that gp order determine by calculating mean of cosine distance}

\subsubsection*{Definitions}

Let \( G_1, G_2, G_3, \dots, G_n \) represent \( n \) different groups of distances.  
The distances in each group \( G_i \) are denoted as:
\[
D_i = \{d_{i1}, d_{i2}, \dots, d_{im_i}\},
\]
where \( m_i \) is the number of elements in group \( G_i \).  

Let:
\[
\{D_1, D_2, \dots, D_n\}
\]
represent the collection of all groups.

After sorting all the distances across the groups into a single list, we check if the order of groups is maintained, i.e., whether all distances in \( G_1 \) are less than those in \( G_2 \), all in \( G_2 \) are less than those in \( G_3 \), and so on.

\subsubsection*{Mathematical Formula for Group-Wise Ranking Accuracy}

\paragraph{Correct Transitions}
A correct transition between two groups \( G_i \) and \( G_j \) (with \( i < j \)) occurs if all elements of \( G_i \) are less than all elements of \( G_j \) after sorting.  

This can be expressed mathematically as:
\[
\text{Correct Transition}(G_i \to G_j) = 
\left[
\forall d_{ik} \in G_i, \forall d_{jl} \in G_j : d_{ik} < d_{jl}
\right], \quad \text{for } i < j.
\]

\paragraph{Total Possible Transitions}
The total number of possible transitions is the number of adjacent group pairs:
\[
T_{\text{total}} = n - 1.
\]

\paragraph{Group-Wise Ranking Accuracy}
The ranking accuracy is the ratio of correct transitions (\( T_{\text{correct}} \)) to total possible transitions (\( T_{\text{total}} \)):
\[
\text{Accuracy} = \frac{T_{\text{correct}}}{T_{\text{total}}}.
\]

\subsubsection*{Step-by-Step Process}

1. \textbf{Sort the Distances:} Sort all the distances from all groups into a single list while keeping track of the group each distance belongs to.

2. \textbf{Check for Correct Transitions:} For each adjacent group pair \( (G_i, G_j) \), check if the condition:
\[
\forall d_{ik} \in G_i, \forall d_{jl} \in G_j : d_{ik} < d_{jl}
\]
holds true.

3. \textbf{Count Correct Transitions:} If the condition holds for a group pair, increment \( T_{\text{correct}} \).

4. \textbf{Compute Accuracy:} Finally, compute the group-wise ranking accuracy as:
\[
\text{Accuracy} = \frac{T_{\text{correct}}}{T_{\text{total}}}.
\]






Curated control dataset: 280
sd: full: 103
exploded full: 40

ds close to sd 10

\begin{table}[H]
  \centering
  \begin{tabular}{@{}llccccc@{}}
  \toprule
  Projection Head & Distance From      & Strong & Sweet & Resize & No Contrast Resize & No Contrast Sweet \\ \midrule
                  & Single Dose Mean   & 92.42      & 91.71     & 92.42      & 91.47                  & 92.42                 \\
  Before          & Control Mean       & 94.55      & 92.65     & 93.84      & \textbf{95.73}                  & 94.55                 \\
                  & Explod Mean        & 82.23      & 82.23     & 92.65     & 83.65                  & 81.52                \\ \midrule
                  & Single Dose Mean   & \textbf{97.39}  & 87.68     & 92.89     & 82.94                  & 83.65                 \\
  After           & Control Mean       & 86.97      & 76.78    & 90.76      & 76.30                 & 82.70                 \\
                  & Explod Mean        & 92.18      & 79.62    & 88.86      & 77.96                  & 82.23                 \\ \bottomrule
  \end{tabular}
  \caption{Cosine distance}
  \label{tab:your_table_label}
\end{table}


\begin{table}[H]
  \centering
  \begin{tabular}{@{}llccccc@{}}
  \toprule
  Projection Head & Distance From      & Strong & Sweet & Resize & No Contrast Resize & No Contrast Sweet \\ \midrule
                  & Single Dose Mean   & 91      & 89.10     & 91.23      & 86.49                  & 90.52                 \\
  Before          & Control Mean       & 87.68      & 86.73     & 88.15      & 82.94                  & 86.73                 \\
                  & Explod Mean        & 81.99      & 81.52     & 92.65      & 83.18                  &   81.52               \\ \midrule
                  & Single Dose Mean   & 90.05      & 81.28     & 91.94      & 77.25                  & 78.67                 \\
  After           & Control Mean       & 82.23      & 76.07     & 84.12      & 76.54                  & 74.17                 \\
                  & Explod Mean        & 90.52      & 78.20    & 94.79      & 75.83                  & 81.75                 \\ \bottomrule
  \end{tabular}
  \caption{Euclidean distance}
  \label{tab:your_table}
\end{table}


\textcolor{red}{check for any pattern in the 97.39 . do other distances depend on time later }

With this costumized metric, we get around 95 if there is slightest seperation for 3 gps.

after projection head: strong seperates perfectly but the problem is since the control is seperated at the middle ordering. Hence we can find the less effective in the middle,

\textcolor{red}{clear seperation between sd and ex ( 2 gp) and max mean difference between them table:}


\begin{table}[H]
  \centering
  \begin{tabular}{@{}llll@{}}
  \toprule
  Distance From & \multicolumn{1}{c}{\begin{tabular}[c]{@{}c@{}}Single Dose\\ Mean\end{tabular}} & \multicolumn{1}{c}{Control Mean} & \multicolumn{1}{c}{Explod Mean} \\ \midrule
  Cosine        & 81.28                                                                           & 86.02                            & 76.78                           \\
  Euclidean     & 83.18                                                                           & 78.44                            & 80.81                           \\ \bottomrule
  \end{tabular}
  \caption{Your table caption here}
  \label{tab:you_label}
\end{table}


\section{Ranking strategy 3: Softmax approach}

This strategy utilizes the before projection head vectors since its linearly seperable for downstream task classification as suggested in original simclr paper.

We will use simclr features since original images are not able to classify 100 percentage as we seen in intermediate evaluation.
1. Train classification model to classify cond7 and sd. 
2. Then do inference on them and take softmax probability as metric for ranking.
below table used 750 epochs, batch:8: all of these value can improve by performing regularizytion and more layers or less layers basically hyper parameter tuning. 
i didn't do it, since i was focused on methodolgy.
Below trained for 750 epochs:batch:8
\begin{table}[H]
  \centering
  \begin{tabular}{@{}lccccc@{}}
  \toprule
  \textbf{Classification} & \textbf{Strong} & \textbf{Sweet} & \textbf{Resize} & \textbf{No Contrast Resize} & \textbf{No Contrast Sweet} \\ \midrule
  Cond7 vs SD             & 97.89           & 99.29          & 98.59           & 98.25                       & 100                        \\
  Cond7 vs Ex             & 97.54               & 96.49              & 98.94               & 100                           &   97.54                        \\
  Ex vs SD             & 95.78               & 98.94              & 95.08               & 92.98                           & 97.54                         \\ \bottomrule
  \end{tabular}
  \caption{Table description goes here.}
  \label{tab:ranking softmax}
\end{table}

\textcolor{red}{Depend on time do ex twenty nine}
  
\textcolor{red}{this questions if strong or No contrast sweet is betterr? because previously strong was better. Now sweet performed well for this task. 
to understand why sweet performs better we need to see crop change and brghtness, contrast change seperatly.}

When we used curated dataset percentage of gp wise accuracy went down where when we used control as the same number of images of sd accuracy is 100. 
it worked maybe because of class balancing.

How do we check which data aug is working?
it should seperate cond7, ex, sd: 100 percentage. because if get 100 that means it is able to make gp of control and sd for sure, from ex mixing.

why are we doing this because every aug can  seperate cond 7 and sd. thats obivious. because there is no other gps. so we have to add new gp as inference to see if 
cond7 and sd still gp together. same principle in kmeans approach.

so, with this approach from cond7 to sd classsification softmax probability score we get mixture of exploded and gray from drug screen.

if we want to avoid that we can train supervised classification to filter out the exploded from there then put them on right side of sd so that we get clean scale.
for that training we can use simclr feature vectors because it gave more accuracy than the original image accuracy.

\textcolor{red}{put the below table in classification and also do classification of ex fourty others because we are usign it for softmax approach} 

\begin{table}[H]
  \centering
  \resizebox{\textwidth}{!}{%
  \begin{tabular}{llccccc}
  \toprule
  \textbf{Augmentation Type}      & \textbf{Metric} & \textbf{Strong} & \textbf{Sweet} & \textbf{Resize} & \textbf{Resize No Contrast} & \textbf{Sweet No Contrast} \\ \midrule
  \multirow{4}{*}{\textbf{Before Projection Head}} 
      & Train Accuracy (\%) & 97.07 & 96.23 & 96.23 & 97.91 & 97.07 \\
      & Train Epoch         & 30 & 41 & 128 & 465 & 485 \\
      & Test Accuracy (\%)  & 95 & 91.67 & 93.33 & 93.33 & 95 \\
      & Test Epoch          & 5 & 5 & 21 & 12 & 17 \\ 
  \bottomrule
  \end{tabular}%
  }
  \caption{Performance metrics for different augmentation strategies before the projection head.}
  \label{tab:augmentation_metrics}
\end{table}

\textcolor{red}{include other classification combinations in table}

Why its not giving 100 because now the problem is harder. we had 2 classes, explod vs all other drug screen,where drug screen look intermediate to explod and sigle dose.
Also we can improve this result by adding more hidden layers instead of just one linear layer and so more. at the moment i didn't since we want to know which data aug works for harder problem.
because in the intermediate evaluation  most of them gave 100 percenatage making it not  evaluator for evaluating inbetween data augs. there we get that orig vs data augs.

\begin{table}[h!]
  \centering
  \caption{Original Image Results}
  \label{tab:original_image_results}
  \begin{tabular}{lcccc}
  \toprule
  \textbf{Metric}         & \textbf{Train Accuracy (\%)} & \textbf{Train Epoch} & \textbf{Test Accuracy (\%)} & \textbf{Test Epoch} \\ \midrule
  \textbf{Original Image} & 84.94                        & 434                   & 78.33                        & 19                  \\ 
  \bottomrule
  \end{tabular}
\end{table}


ordered images can found in the below gdrive links for both mixed and cleaned order for ds and ex.
show them ordered images for 80 percentage gp wise accuracy ie only seperated 2 gps, show them its worse.

\textcolor{red}{its better to not use exploded and harm people sentence from intro and data intro just say debris, because cond10 and exploded have debris. cond10 
have debris even without drug so debris can happen without drug.it is probabliy because its cultivated in lab where these things happen.} 

\chapter{Conclusion}

\section{Final Evaluation}
Now out of the 4 methodologies which methodology is more reliable and consistant?
I picked 22 sinle dose  similar images from drug screening images (Note: This final Evaluation is not relaible in the sense that the picked images are not verified by any biology expert that those are the most similar images to single dose images from drug screening. so eventhough this final evaluation is not scientific its better than nothing). Idea is that we will pick the best performed data augmentation type/category from each methodology and check whether these images are still positioned in the range of the single dose image range.
\begin{table}[H]
	\centering
	\small
	\begin{tabular}{@{}lc@{}}
	\toprule
	\textbf{Ranking Strategy}                                                                       & \textbf{Out of single dose range} \\ \midrule
	Softmax: cond7 vs sd                                                                            & 0                   \\ 
	Softmax: cond7 vs Ex                                                                            & 0                   \\ 
	\begin{tabular}[c]{@{}l@{}}K-means: After cosine strong\\ using single dose mean\end{tabular}   & 12   controlil kedakkane              \\ 
	PCA: strong cosine distance                                                                     & 12 controlil kedakkane                   \\ 
	PCA: Euclidean strong                                                                           & 11  controlil               \\ \bottomrule
	\end{tabular}
	\caption{Performance metrics across different ranking strategies.}
	\label{tab:ranking_strategies}
\end{table}



	
	Conclusions - yes, what are they? What is the outcome. Important - it is fine to say that despite your best efforts the methods do not seem to
	 deliver useful results. Perhaps they do not beat the baseline, this is very common in research and is perfectly fine, correct and useful to admit.
	  Or they actually seem to perform randomly and are not reliable. This may be due to not enough data and is also ok to say, admit and point out. Or 
	  they do something but you are not actually sure, how to use for the final objective or ranking. This is also ok and you shall explain why you are 
	  not convinced. Perhaps the ranking is simply a badly formulated problem and it should not be approached this way at all. This is also ok to say and 
	  be open about it. Simply, even if it "does not work" it is still VERY useful if we understand what you have done and why you think it does not work.	
How do you know which ranking strategy is better visually? inference from ds close tosd for the worked one from each strategy and look which one works 
better?

References - clean these. Correct references shall have not only the names and year but also the publication venue (journal, conference, arxiv... etc.)

\section{Future research direction}

\begin{enumerate}
    \item Hybrid: Include human-interpretable features in the unsupervised learning features combined effect.
    \item Weakly supervised DINO transformer-based approach. Future work: Later, other models such as masked autoencoders and DINO will be explored, depending on the available time.
	Why we would like to try other models? Because SimCLR demands larger batch size and more data for better performance which we don't have.
    \item Weakly supervised SimCLR approach.
    \item Other distance-based approaches.
    \item kl density estimation probability
    \item Image size for simclr 96 vs 256 vs 512 as well as original images 96 vs 256 vs 512 vs even original size 2054?
    \item other architectures. Resnet vs unet in Simclr
    \item 3 channel vs 1 channel (mis alignment probelm or compuatoinal faster?)
    \item Batch size 16 vs 64 vs 128 vs 256
    \item \textcolor{red}{after projectoin add l2 norm so that it will exactly like the loss fn gives good cosine sim} 

\end{enumerate}



\backmatter
%%%%%%%%%%%%%%%%%%%
%% Appendices
%%%%%%%%%%%%%%%%%%%
\appendix
\chapter{Appendix}\label{appendix}
% Add appendix content here

Derivation of K-Means Clustering using Euclidean Distance and Mean

Objective Function
The k-means algorithm aims to minimize the total squared Euclidean distance between data points and their assigned cluster centroids. The objective function is:

\[
J = \sum_{i=1}^{n} \sum_{k=1}^{K} r_{ik} \| \mathbf{x}_i - \boldsymbol{\mu}_k \|^2
\]

where:
\begin{itemize}
    \item \( n \): Number of data points,
    \item \( K \): Number of clusters,
    \item \( \mathbf{x}_i \): The \( i \)-th data point,
    \item \( \boldsymbol{\mu}_k \): The centroid of the \( k \)-th cluster,
    \item \( r_{ik} \): Binary indicator; \( r_{ik} = 1 \) if \( \mathbf{x}_i \) belongs to cluster \( k \), otherwise \( r_{ik} = 0 \).
\end{itemize}

\subsection*{Cluster Assignment Step}
For a fixed set of centroids \( \{ \boldsymbol{\mu}_k \}_{k=1}^K \), assign each data point \( \mathbf{x}_i \) to the nearest centroid. This minimizes:

\[
r_{ik} =
\begin{cases} 
1, & \text{if } k = \arg\min_{j} \| \mathbf{x}_i - \boldsymbol{\mu}_j \|^2, \\
0, & \text{otherwise.}
\end{cases}
\]

\subsection*{Centroid Update Step}
For a fixed cluster assignment \( \{ r_{ik} \} \), minimize \( J \) with respect to the centroids \( \{ \boldsymbol{\mu}_k \} \):

\[
J = \sum_{i=1}^{n} \sum_{k=1}^{K} r_{ik} \| \mathbf{x}_i - \boldsymbol{\mu}_k \|^2
\]

Focus on a single cluster \( k \). The term involving \( \boldsymbol{\mu}_k \) is:

\[
\sum_{i=1}^{n} r_{ik} \| \mathbf{x}_i - \boldsymbol{\mu}_k \|^2
= \sum_{i=1}^{n} r_{ik} \left( \mathbf{x}_i^\top \mathbf{x}_i - 2 \mathbf{x}_i^\top \boldsymbol{\mu}_k + \boldsymbol{\mu}_k^\top \boldsymbol{\mu}_k \right)
\]

Take the derivative with respect to \( \boldsymbol{\mu}_k \) and set it to zero:

\[
\frac{\partial}{\partial \boldsymbol{\mu}_k} \sum_{i=1}^{n} r_{ik} \| \mathbf{x}_i - \boldsymbol{\mu}_k \|^2 =
-2 \sum_{i=1}^{n} r_{ik} \mathbf{x}_i + 2 \sum_{i=1}^{n} r_{ik} \boldsymbol{\mu}_k = 0
\]

Simplify:

\[
\sum_{i=1}^{n} r_{ik} \mathbf{x}_i = \sum_{i=1}^{n} r_{ik} \boldsymbol{\mu}_k
\]

Factor out \( \boldsymbol{\mu}_k \):

\[
\boldsymbol{\mu}_k = \frac{\sum_{i=1}^{n} r_{ik} \mathbf{x}_i}{\sum_{i=1}^{n} r_{ik}}
\]

This is the mean of the points in cluster \( k \).

\subsection*{Algorithm Summary}
The k-means algorithm alternates between the following two steps until convergence:

\begin{enumerate}
    \item \textbf{Cluster Assignment Step}: Assign each point \( \mathbf{x}_i \) to the nearest cluster:
    \[
    r_{ik} =
    \begin{cases} 
    1, & \text{if } k = \arg\min_{j} \| \mathbf{x}_i - \boldsymbol{\mu}_j \|^2, \\
    0, & \text{otherwise.}
    \end{cases}
    \]
    \item \textbf{Centroid Update Step}: Update the centroid of each cluster as the mean of its assigned points:
    \[
    \boldsymbol{\mu}_k = \frac{\sum_{i=1}^{n} r_{ik} \mathbf{x}_i}{\sum_{i=1}^{n} r_{ik}}
    \]
\end{enumerate}

This iterative process continues until the assignments \( r_{ik} \) and centroids \( \boldsymbol{\mu}_k \) no longer change or the change is below a threshold.

\subsection*{Cosine distance}

\section*{Normalization}

To calculate the cosine distance, we first \textbf{normalize} all data points and centroids. Suppose the normalized data points and centroids are \( \mathbf{x}_i \) and \( \mathbf{c}_k \), respectively, then:

\[
\|\mathbf{x}_i\| = 1 \quad \text{and} \quad \|\mathbf{c}_k\| = 1.
\]

This ensures all vectors are on the unit sphere. The cosine similarity between two vectors \( \mathbf{x}_i \) and \( \mathbf{c}_k \) is defined as:

\[
\text{cosine similarity} = \frac{\mathbf{x}_i^\top \mathbf{c}_k}{\|\mathbf{x}_i\| \|\mathbf{c}_k\|}
\]

Since \( \|\mathbf{x}_i\| = 1 \) and \( \|\mathbf{c}_k\| = 1 \), we substitute these values into the equation:

\[
\text{cosine similarity} = \frac{\mathbf{x}_i^\top \mathbf{c}_k}{1 \times 1}
\]

This simplifies to:

\[
\text{cosine similarity} = \mathbf{x}_i^\top \mathbf{c}_k.
\]

Thus, the \textbf{cosine distance} becomes:
\[
\text{cosine distance} = 1 - \mathbf{x}_i^\top \mathbf{c}_k.
\]



\section*{Relating Cosine Distance to Euclidean Distance}

For normalized vectors, we derive the relationship between \textbf{Euclidean distance} and \textbf{cosine distance}. The squared Euclidean distance between a data point \( \mathbf{x}_i \) and a centroid \( \mathbf{c}_k \) is:
\[
\|\mathbf{x}_i - \mathbf{c}_k\|^2 = \sum_{j} (x_{ij} - c_{kj})^2.
\]
Expanding this:
\[
\|\mathbf{x}_i - \mathbf{c}_k\|^2 = \|\mathbf{x}_i\|^2 + \|\mathbf{c}_k\|^2 - 2 \mathbf{x}_i^\top \mathbf{c}_k.
\]
Since \( \|\mathbf{x}_i\| = 1 \) and \( \|\mathbf{c}_k\| = 1 \), we get:
\[
\|\mathbf{x}_i - \mathbf{c}_k\|^2 = 1 + 1 - 2 \mathbf{x}_i^\top \mathbf{c}_k.
\]
Simplify:
\[
\|\mathbf{x}_i - \mathbf{c}_k\|^2 = 2(1 - \mathbf{x}_i^\top \mathbf{c}_k).
\]
Thus, for normalized vectors, the Euclidean distance is proportional to the cosine distance:
\[
\|\mathbf{x}_i - \mathbf{c}_k\|^2 = 2 \cdot \text{cosine distance}.
\]
Rearranging to express the cosine distance:
\[
\text{cosine distance} = \frac{\|\mathbf{x}_i - \mathbf{c}_k\|^2}{2}.
\]




\section*{Objective Function}

The k-means algorithm with cosine distance aims to minimize the cosine distance between data points \( \mathbf{x}_i \) and their assigned cluster centroids \( \mathbf{c}_k \). The objective function is:

\[
J = \sum_{i=1}^{n} \sum_{k=1}^{K} r_{ik} \left( 1 - \frac{\mathbf{x}_i^\top \mathbf{c}_k}{\|\mathbf{x}_i\| \|\mathbf{c}_k\|} \right),
\]
where:
\begin{itemize}
    \item \( n \): Number of data points,
    \item \( K \): Number of clusters,
    \item \( \mathbf{x}_i \): \( i \)-th data point,
    \item \( \mathbf{c}_k \): Centroid of cluster \( k \) (normalized to unit length),
    \item \( r_{ik} \): Binary indicator; \( r_{ik} = 1 \) if \( \mathbf{x}_i \) belongs to cluster \( k \), otherwise \( r_{ik} = 0 \).
\end{itemize}

\section*{Objective Function in Terms of Euclidean Distance}

Using the above result, the k-means objective function with cosine distance:
\[
J = \sum_{i=1}^{n} \sum_{k=1}^{K} r_{ik} \left( 1 - \mathbf{x}_i^\top \mathbf{c}_k \right)
\]
can be rewritten in terms of Euclidean distance:
\[
J = \sum_{i=1}^{n} \sum_{k=1}^{K} r_{ik} \frac{\|\mathbf{x}_i - \mathbf{c}_k\|^2}{2}.
\]
Here, the factor \( \frac{1}{2} \) accounts for the scaling difference.

\section*{Cluster Assignment Step}

For a fixed set of centroids \( \{ \mathbf{c}_k \}_{k=1}^K \), assign each data point \( \mathbf{x}_i \) to the nearest cluster based on the \textbf{cosine similarity} (or equivalently, minimize cosine distance):
\[
r_{ik} =
\begin{cases}
1, & \text{if } k = \arg\max_{j} \mathbf{x}_i^\top \mathbf{c}_j, \\
0, & \text{otherwise.}
\end{cases}
\]
\subsection*{Centroid Update Step for Cosine Distance}
For a fixed cluster assignment \( \{ r_{ik} \} \), minimize \( J \) with respect to the centroids \( \{ \mathbf{c}_k \} \):

\[
J = \sum_{i=1}^{n} \sum_{k=1}^{K} r_{ik} \frac{\|\mathbf{x}_i - \mathbf{c}_k\|^2}{2}
\]

Focus on a single cluster \( k \). The term involving \( \mathbf{c}_k \) is:

\[
\sum_{i=1}^{n} r_{ik} \frac{\|\mathbf{x}_i - \mathbf{c}_k\|^2}{2}
= \sum_{i=1}^{n} r_{ik} \frac{1}{2} \left( \|\mathbf{x}_i\|^2 + \|\mathbf{c}_k\|^2 - 2 \mathbf{x}_i^\top \mathbf{c}_k \right)
\]

Since the vectors are normalized, \( \|\mathbf{x}_i\| = 1 \) and \( \|\mathbf{c}_k\| = 1 \), we have:

\[
\sum_{i=1}^{n} r_{ik} \frac{1}{2} \left( 1 + 1 - 2 \mathbf{x}_i^\top \mathbf{c}_k \right)
= \sum_{i=1}^{n} r_{ik} \left( 1 - \mathbf{x}_i^\top \mathbf{c}_k \right)
\]

Now, take the derivative of the above with respect to \( \mathbf{c}_k \) and set it to zero:

\[
\frac{\partial}{\partial \mathbf{c}_k} \sum_{i=1}^{n} r_{ik} \left( 1 - \mathbf{x}_i^\top \mathbf{c}_k \right)
= \sum_{i=1}^{n} r_{ik} \mathbf{x}_i = \sum_{i=1}^{n} r_{ik} \mathbf{c}_k
\]

Simplify:

\[
\sum_{i=1}^{n} r_{ik} \mathbf{x}_i = \sum_{i=1}^{n} r_{ik} \mathbf{c}_k
\]

Factor out \( \mathbf{c}_k \):

\[
\mathbf{c}_k = \frac{\sum_{i=1}^{n} r_{ik} \mathbf{x}_i}{\sum_{i=1}^{n} r_{ik}}
\]

This is the mean of the normalized points in cluster \( k \).

\section*{Centroid Update Step}

For a fixed cluster assignment \( \{ r_{ik} \} \), update the centroids \( \{ \mathbf{c}_k \} \) as the \textbf{normalized mean} of all data points assigned to cluster \( k \):

\[
\mathbf{c}_k = \frac{\sum_{i=1}^{n} r_{ik} \mathbf{x}_i}{\left\| \sum_{i=1}^{n} r_{ik} \mathbf{x}_i \right\|}.
\]

\section*{Algorithm Summary}

The k-means algorithm with cosine distance alternates between two steps until convergence:

\begin{enumerate}
    \item \textbf{Cluster Assignment Step}: Assign each point \( \mathbf{x}_i \) to the cluster with the highest cosine similarity:
    \[
    r_{ik} =
    \begin{cases}
    1, & \text{if } k = \arg\max_{j} \mathbf{x}_i^\top \mathbf{c}_j, \\
    0, & \text{otherwise.}
    \end{cases}
    \]

    \item \textbf{Centroid Update Step}: Update the centroid of each cluster as the normalized mean of its assigned points:
    \[
    \mathbf{c}_k = \frac{\sum_{i=1}^{n} r_{ik} \mathbf{x}_i}{\left\| \sum_{i=1}^{n} r_{ik} \mathbf{x}_i \right\|}.
    \]
\end{enumerate}

\section*{Conclusion}

By normalizing the data points and centroids, the k-means clustering objective can be expressed in terms of both cosine distance and Euclidean distance. The equivalence:
\[
\|\mathbf{x}_i - \mathbf{c}_k\|^2 = 2(1 - \mathbf{x}_i^\top \mathbf{c}_k)
\]
enables seamless interpretation and implementation in the algorithm.

%%%%%%%%%%%%%%%%%%%
%% create bibliography
%%%%%%%%%%%%%%%%%%%
\cleardoublepage
\phantomsection
\addcontentsline{toc}{chapter}{Literature}
\printbibliography

%%%%%%%%%%%%%%%%%%%
%% declaration on oath
%%%%%%%%%%%%%%%%%%%
\addchap{Declaration on oath}

I hereby certify that I have written my master thesis independently and have not yet submitted it for examination purposes elsewhere. All sources and aids used are listed, literal and meaningful quotations have been marked as such.

\vspace{20pt}
\begin{flushright}
$\overline{~~~~~~~~~~~~~~~~~\mbox{\ShowBaAuthor, \SubmitDate}~~~~~~~~~~~~~~~~~}$
\end{flushright}

\addchap{Consent to plagiarism check}

I hereby agree that my submitted work may be sent to PlagScan (www.plagscan.com) in digital form for the purpose of checking for plagiarism and that it may be temporarily (max. 5~years) stored in the database maintained by PlagScan as well as personal data which are part of this work may be stored there.

\begin{small}
Consent is voluntary. Without this consent, the plagiarism check cannot be prevented by removing all personal data and protecting the copyright requirements. Consent to the storage and use of personal data may be revoked at any time by notifying the faculty.
\end{small}

\vspace{20pt}
\begin{flushright}
$\overline{~~~~~~~~~~~~~~~~~\mbox{\ShowBaAuthor, \SubmitDate}~~~~~~~~~~~~~~~~~}$
\end{flushright}
\end{document}
