\chapter{Research Objective and Questions}\label{ch:Research Objectives and Questions}
\section*{Research objective}
This thesis aims to assess drug efficacy by ranking different drug combinations and concentrations on lab-cultivated pancreatic 3D tumor tissue models. 
The ranking is based on features extracted from bright-field microscopy images of these tumor tissue models using
SimCLR self supervised learning. The primary challenge lies in learning the efficient features of alterations induced in these
tumor tissue models by the impact of drug application over a period of time.

Other main challenges in the ranking strategy are the lack of groundtruth labels , except we only know that drug efficacy is zero for untreated (control) images. Hence supervised learning is not possible and also lack of data. Since we lack ground-truth labels, ranking to get exact efficacy value will be out of focus. Instead, the focus will be on formulating ranking strategies to order the drug screening images relative to the various possible transitions between classes, such as untreated to single dose to exploded images, or alternative sequences like single dose to untreated to exploded, or exploded to untreated to single dose etc. But the lack of ground truth labels also makes the problem harder since we don't have a standard evaluation method to assess whether the ranking strategy is effective.
\section*{Research Questions}

1. Can we learn latent features that capture the alterations induced in tumor tissue models by drug application over a period of time from bright-field microscopy images?

2. Will these features effectively establish a relative ranking assessment of drug efficacy?

3. What methodologies and frameworks can be employed to extract and learn these hidden representations efficiently?

4. What could be reasonable metrics, such as L2 loss or cosine similarity, to support the relative assessment of drug efficacy?

