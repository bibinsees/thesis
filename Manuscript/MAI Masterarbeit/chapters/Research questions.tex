\chapter{Research Objective and Questions}\label{ch:Research Objectives and Questions}
\section*{Research objective}
This thesis aims to assess drug efficacy by ranking different drug combinations and concentrations on lab-cultivated pancreatic tumor tissue models. 
The ranking is based on features extracted from bright-field microscopy images of these tumor tissue models using
SimCLR self supervised learning. Since we lack ground truth labels, ranking to get exact efficacy value and evaluation of our ranking system to exact drug efficacy metric 
 will be out of focus, instead we rank the images as a transition from untreated to single dose images or untreated to explod images. This is our objective.  


 \textcolor{red}{one more thing. You had a master thesis proposal in the beginning which actually listed the research questions as well as the methodology. 
 Your final manuscript shall reflect it and follow it. When you diverted from the original proposal, you shall explain why you diverted.} 


\textcolor{red}{My plan is to rewrite the introduction to include why we use SimCLR as self supervised learning for learning representation by including :
 basically in simclr the loss function is defined such a way that it reduce the cosine distance between the augmentations derived from one image  and no 
 penalisation for pushing or no pushing away of augmentation from one image to augmentation from another image. this is perfect in our case where 
 we don't want to push away augmented images derived from different images, especially when we don't know if those images have same drug efficay or not.}

\textcolor{red}{Doubt: Should I need to answer the below questions here it self or can I answer as i write through out the the whole section?}

\section*{Research Questions}
1. Can unsupervised learning be used to tackle this ranking problem, given the lack of labels as our primary challenge?

2. Can we learn latent features that capture the alterations induced in tumor tissue models by drug application from bright-field microscopy images?

3. Will these features effectively establish a ranking of drug efficacy?

4. What methodologies and frameworks can be employed to extract and learn these hidden representations efficiently?

5. What could be reasonable metrics, such as L2 loss or cosine similarity, to support the relative assessment of drug efficacy?

6. Do we actually need to learn latent features to establish an effective ranking, or is it possible to achieve this directly using the images themselves?

8. What should be done if the same drug concentration has different effects on cancer cells across different experiments? (For instance, 
we may not know whether the observed differences are due to real efficacy or visual effects, such as debris.)


9. How do we deal with the position change in the image collection of day 10 if we want to use prediction 
model as one of our ranking strategy?

10. How do we incorporate the fact about brightness/blur change in the image collection due to environmental or microscope 
variations?

11. Will strong data augmentation on medical grayscale images improve performance, or will it degrade it due to sensitivity to the original distribution?

12. If yes, If we reduce the intensity of augmentation, will it still be able to learn the features effectively?


13. Does data augmentation function as a tool for learning invariant features, or does it primarily act as a mechanism to increase the sample size,
 enabling the model to see all possible distributions?



