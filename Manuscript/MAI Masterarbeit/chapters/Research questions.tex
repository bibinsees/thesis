\chapter{Research Objective and Questions}\label{ch:Research Objectives and Questions}
\section*{Research Objective}  
This thesis aims to assess drug efficacy by ranking different drug combinations and concentrations on lab-cultivated pancreatic 3D tumor tissue models.  
The ranking is based on features extracted from bright-field microscopy images of these tumor tissue models. The primary challenge lies in efficiently learning the features of alterations induced in these tumor tissue models by the impact of drug application over a period of time.  

Another significant challenge in the ranking strategy is the absence of ground truth labels. The only available information is that drug efficacy is zero for untreated (control) images. Consequently, supervised learning is not possible, and the limited amount of data further complicates the task. Therefore, SimCLR, a self-supervised learning method \cite{chen2020simple}, is used to learn latent representations of the images. Due to the absence of ground truth labels, determining an exact efficacy value through ranking will be the out of focus. Instead, the focus will be placed on formulating ranking strategies to order the drug screening images relative to the various possible transitions between classes, such as untreated to single dose to exploded images, or alternative sequences like single dose to untreated to exploded, or exploded to untreated to single dose, etc as explained in the chapter \ref{ch:Methodology for Ranking}. However, the lack of ground truth labels also complicates the problem, as a standard evaluation method to assess the effectiveness of the ranking strategy is unavailable.  
\section*{Research Questions}

1. Can we learn latent features that capture the alterations induced in tumor tissue models by drug application over a period of time from bright-field microscopy images?

2. Will these features effectively establish a relative ranking assessment of drug efficacy?

3. What methodologies and frameworks can be employed to extract and learn these hidden representations efficiently?

4. What could be reasonable metrics, such as L2 loss or cosine similarity, to support the relative assessment of drug efficacy?

