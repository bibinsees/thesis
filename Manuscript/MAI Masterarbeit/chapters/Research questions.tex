\chapter{Research questions}\label{ch:Research questions}
This thesis aims to assess drug efficacy by ranking different drug combinations and concentrations. 
The ranking is based on features extracted from bright-field microscopy images of  three-dimensional tumor tissue models using
representation learning. 

1. Can we learn latent features that capture the alterations 
induced in three-dimensional tumor tissue models by drug application over a period of time from bright-field microscopy images?

2.  Will these features can effectively establish a ranking of drug efficacy?

3. What methodologies and frameworks can be employed to extract and learn these
 hidden representations efficiently?

4. What could be reasonable metrics, such as L2 loss or cosine similarity, for 
supporting the relative assessment of drug efficacy?

5. can unsupervised learning be used to tackle this ranking problem since lack of label is 
our main challenge?

6. How do we deal with the position change in the image collection of day 10 if we want to use prediction 
model as one of our ranking startegy?

7. How do we incorporate the fact about brightness/blur change in the image collection due to environmental or microscope 
variations?

8. how do we deal with the insufficient data problem?

9. Do we actually need to learn these latent features to establish effective ranking ? or do would it be possible to do it directly using images it self?

10. If we use simclr then is our dataset sensitive to strong data augementation in that case if we reduce the intensity will it able to learn the features?

11. What to do if same concentraion drug have diferent effect on the cancer cell in different experimant? ( we don't know if its visual effect or real efficay effect)
but for sure there is visual effect in the sense debris. 

12. Does medical gray scale image will give better performance with  strong data augementation or will it be worse since its sensitive to original distribution.
13. Does Data augmentation works as a learning tool to learn the invariant features or it is act as increasing the sample size so that it sees all possible distribution
 that can happen to the same image groups?

 
