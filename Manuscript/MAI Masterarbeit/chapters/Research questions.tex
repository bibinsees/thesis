\chapter{Research Objective and Questions}\label{ch:Research Objectives and Questions}
\section*{Research objective}
This thesis aims to assess drug efficacy by ranking different drug combinations and concentrations on lab-cultivated pancreatic tumor tissue models. 
The ranking is based on features extracted from bright-field microscopy images of these tumor tissue models using
SimCLR self supervised learning. Since we lack ground truth labels, ranking to get exact efficacy value will be out of focus, instead we rank the images as a transition from
 untreated to single dose images or untreated to explod images. This is our objective.  


 \textcolor{red}{one more thing. You had a master thesis proposal in the beginning which actually listed the research questions as well as the methodology. 
 Your final manuscript shall reflect it and follow it. When you diverted from the original proposal, you shall explain why you diverted.} 


\textcolor{red}{My plan is to rewrite the introduction to include why we use SimCLR as self supervised learning for learning representation by including :
 basically in simclr the loss function is defined such a way that it reduce the cosine distance between the augmentations derived from one image  and no 
 penalisation for pushing or no pushing away of augmentation from one image to augmentation from another image. this is perfect in our case where 
 we don't want to push away augmented images derived from different images, especially when we don't know if those images have same drug efficay or not.}

\section*{Research Questions}

1. Can we learn latent features that capture the alterations induced in tumor tissue models by drug application over a period of time from bright-field microscopy images?

2. Will these features effectively establish a relative ranking assessment of drug efficacy?

3. What methodologies and frameworks can be employed to extract and learn these hidden representations efficiently?

4. What could be reasonable metrics, such as L2 loss or cosine similarity, to support the relative assessment of drug efficacy?




