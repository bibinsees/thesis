\chapter{Methodology}\label{ch: Methodology}
The goal is to leverage representation learning of bright-field microscopy images to develop a ranking/ordering scale (1 to n) for these images. 

\begin{enumerate}
  \item \textbf{Step 1:} Create a latent space representation of each image using contrastive learning techniques such as SimCLR, masked autoencoder, or any other self-supervised architectures such as DINO that can effectively help learn the efficient features of alterations induced in three-dimensional tumor tissue models by the impact of drug application over a period of time.
  
  \item \textbf{Step 2:} Train a time series prediction model exclusively on the representations of untreated images from Day 7 to Day 10 to predict the representation of the Day 10 image.
  
  \item \textbf{Step 3:} Perform inference on the representations of test images, which include untreated, clinically recommended, and drug screening images.
\end{enumerate}

Since the time series model is trained solely on the representations of untreated images, the inference loss/metric (i.e., the difference between the predicted and actual Day 10 image representations) will be very small for untreated images. Conversely, the inference loss/metric will increase for treated images as their representations deviate from those of untreated images. This inference loss/metric will be used as the feature for the ranking/order scale, where the initial images will start with untreated images that have very small inference loss/metric, and the scale will end with images having high inference loss/metric in ascending order. Determining a reasonable inference loss/metric will be one of the research problems to tackle.
