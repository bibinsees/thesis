\chapter{Experimental Results and Discussion} \label{chapter:experimental results and Discussion} 

\section{3 channel vs 1 channel}
\section{Unet vs resnet}



\section{Intermediate evaluation}
\subsection{Classification} \label{subsection:classification}

\subsection{Clustering} \label{subsection:kmeans}

\subsection{Data augmentation  day 7 to day 10 distance evaluation} \label{subsection:day7today10distance}
basic evaluations for ranking:
1.
whether it learned to be invariant the position change:
do flipps and calculate the cosine distance from control to treated flipped versions. ( i don't expect it learn to the change in center of position, 
if it learns good we can say center crop have some effect maybe? but not for the edge one?)
2. 
shape invariant:control to all single dose should be almost same cosine distance.

it also applicable to time prediction and reconstruction ranking evaluation and also from kmeans centriod approach.



\section{Ranking Evaluation}

\subsection{Using CAE} \label{subection:Using CAE}
\subsubsection{Day 7 to 7 reconstruction} \label{subsubection:Day 7 to 7 reconstruction}
Original images: Unfortunately, the inference loss/metric will ve same beacuse control day 10 and treatd looks same . I was dumb enough to do that. thats why i need  day 7 to day 7 reconstruction model.
\subsubsection{Day 7 to 10 prediction} \label{subsubection:Day 7 to 10 prediction}
\paragraph{Day 10 prediction model} \label{paragraph:Day 10 prediction model}
\paragraph{Delta prediction model} \label{paragraph:delta prediction}
\subsection{K means centroid approach} \label{subsection:kmeanscentroidapproach}
\subsection{Softmax approach} \label{subsection:kmeanscentroid}




\lipsum[3]
\begin{table}[H]
	\begin{center}
		\def\arraystretch{2}
		\begin{tabular}{p{0.155\linewidth}|p{0.15\linewidth}|p{0.145\linewidth}|p{0.14\linewidth}|p{0.14\linewidth}|p{0.15\linewidth}}
			\textbf{A}& ...	&...	&... &...  \\
			\hline \textbf{B} &...&...&... &... \\
			\hline \textbf{C} & ... &... &... &... &... \\
			\hline \textbf{D} & ... &...&... &... &... \\
			\hline \textbf{E} & ... &... &... &... &... \\
			\hline \textbf{F} & - &- &- &$\surd$  &$\surd$  \\
			\hline \textbf{G} & - & -& - & $l \leq x$ & $11 \leq l \leq x$\\
		\end{tabular}
	\end{center}
	\caption{Cool table}
	\label{tab:cool_table}
\end{table}