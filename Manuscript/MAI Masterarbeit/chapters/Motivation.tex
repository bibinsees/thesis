\chapter{Motivation} \label{chapter:Motivation} 

We assess the efficacy of the drug by comparing the changes it induces in the bright-field microscopy images over a period of time. 
The current methods to differentiate these changes involve studying the alterations from day 7 to day 10. These changes are typically
observed in three main parameters: 

\begin{enumerate}
  \item Size/Area
  \item Circularity/Diameter/Perimeter
  \item Pixel intensity or color change
\end{enumerate}

These parameters serve as human-interpretable metrics for assessing the efficacy of the drug. However, these manual methods are inherently limited to
observable features, which may overlook subtle or complex patterns within the data.


Representation learning techniques, such as those based on deep learning, can be employed to extract features that are not immediately interpretable by 
humans. These methods leverage neural networks to learn high-dimensional feature representations directly from the images, capturing intricate patterns, 
relationships, and variations that may correspond to biological phenomena. Furthermore, these learned representations enable a standardized approach to analysis. Unlike manual assessments, which can vary due to human subjectivity,
representation learning models produce consistent results once trained. 

This is the focus of my master thesis, where I take on the crucial role of developing and applying representation learning techniques to uncover these
hidden patterns and standardize the analysis process.

\let\cleardoublepage\clearpage

