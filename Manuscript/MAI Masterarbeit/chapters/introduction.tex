\chapter{Introduction}\label{ch:intro}

Pancreatic tumor presents a significant challenge in terms of treatment due to its heterogeneous nature and the mutations that occur during its progression within the human body. Clinicians rely on case studies, human trials, and their own expertise gained from past patient treatments to select drugs for new patients. However, this approach is often based on trial and error, with varying outcomes. Patients may experience either successful treatment or severe side effects such as hair loss and damage to other organs. Since each patient's tumor cells exhibit unique characteristics influenced by factors such as age and genetics, treatments that have worked for one patient may not be effective for another. Consequently, clinicians may need to change the prescribed drugs or try different combinations, which can lead to delays and increased risks for the patient, including mortality.

In light of these challenges researchers at Fraunhofer Translational Center for Regenerative Therapy TLZ-RT Wuerzburg,  propose a vision for the future: cultivating multiple three-dimensional tumor tissue models for each patients in the lab using biopsy samples and studying the efficacy of drugs on these three-dimensional tumor tissue models first.\textit{(Note: In this thesis, "3D tumor tissue models or tumor tissue models" refers to physical, lab-grown tissues and not computational or AI models.)} By conducting drug development experiments and analyses on these tissue models, they aim to find the optimum or best drug combination tailored to each patient's specific tumor characteristics. This approach can not only minimize direct side effects on human patients and reduce the time needed to select the most effective personalized treatment, thereby decreasing the risk of that patient's mortality, but also significantly reduce the cost and time of preclinical testing in the drug development process. Ultimately, these information obtained from drug efficacy assessment experiments can inform clinicians' decisions, enabling them to select the most effective drug combination before administering it to the patient.

As a proof of concept, The Fraunhofer TLZ-RT Wuerzburg laboratory utilizes a modular dual-arm robot-based system \cite{Dembski2023Establishing}, equipped with incubators and bioreactors (see Figure \ref{fig:platform} and Figure \ref{fig:Robot}) under physiological conditions to study drug efficacy for the long-term culture of these three-dimensional tumor tissue models. One advantage of this platform is its ability to capture bright-field microscopy images of 3D tumor tissue models using a customized microscope setup integrated into the robotic platform, offering flexibility in image acquisition according to experimental needs.

\begin{figure}[H]
  \centering
  \includegraphics[scale=0.4]{figures/platform setup.png} % Adjust the width as needed
  \caption{Robo platform}
  \label{fig:platform}
\end{figure}



\begin{figure}[H]
  \centering
  \includegraphics[scale=0.4]{figures/robot.png} % Adjust the scale factor as needed
  \caption{Dual-arm robot}
  \label{fig:Robot}
\end{figure}

Although the vision for the future is to simulate the identical interaction environment of drugs with tumor cells as it occurs in the human body, current technology has not yet achieved this. The current three-dimensional tumor tissue models developed in the lab do not fully resemble real pancreatic tumor cells found in the human body. These 3D tumor tissue models only contain pure tumor tissues, whereas real human pancreatic tumor cells exist within a complex microenvironment comprising tumor cells, blood vessels, other tissues, and various cell types.
Fortunately, if human body tumor cells can be replicated in the lab in the future, the techniques currently used to study bright-field microscopy images will still be applicable. However, the fact that bright-field microscopy images are two-dimensional limits the ability to perform a comprehensive analysis of the drug's impact on the entire 3D structure of the cultivated tumor tissue models. Despite this limitation, this research serves as a valuable starting point for studying drug efficacy in a controlled environment. 

Alternatives to bright-field microscopy images include 3D fluorescence microscopy and luminescent cytotoxicity assays. However, both methods are invasive. Fluorescent molecules tend to generate reactive chemical species under illumination, enhancing phototoxic effects. This chemical reaction with the 3D tumor tissue model may alter its structure, making it not suitable to isolate the drug's effect over time. Similarly, luminescent cytotoxicity assays result in a dead culture, rendering them unsuitable for longitudinal studies. Additionally, both methods require removing the well plate from the isolated culture environment for extended periods, making the samples susceptible to external environmental factors. For instance, in fluorescence microscopy, cells are particularly vulnerable to phototoxicity from short wavelength light. In contrast, bright-field microscopy images are non-invasive, allowing continuous culture and the possibility of creating time series of images to study dynamic changes. Therefore, we rely on bright-field microscopy images to study the time-evolutionary effects of drugs.

\section{Laboratory Setup}
\label{sec:lab-setup}
3D tumor tissue models are cultured in well plates containing 96 wells, each providing a nutrient medium that allows them to maintain their tissue-specific functions in vitro. Although each plate can yield 96 pure 3D tumor tissue models, the edge effect is accounted for, where outer wells may be exposed to variable conditions such as temperature fluctuations, increased evaporation rates, and other environmental factors. Consequently, we restrict our analysis to the 60 inner wells per plate as in figure \ref{fig:Wellplate}, adhering to standard procedures to ensure consistent and reliable experimental data. 

\begin{figure}[H]
  \centering
  \includegraphics[width=0.9\linewidth]{figures/WellPlate.png} % Adjust the width as needed
  \caption{A well plate containing 96 wells where rows A, H and columns 1, 2 are excluded due to edge effects.}
  \label{fig:Wellplate}
\end{figure}





Based on the drug concentration applied to 3D tumor tissue models, the bright-field microscopy images we capture can be categorized into three:

Images of

\begin{enumerate}
  \item Control (0 percentage drug applied)
        \begin{itemize}
            \item For easiness, we refer to this category as ``Untreated''
        \end{itemize}
  
  \item Single concentration (theoretically recommended single concentration of drug treatment)
        \begin{itemize}
            \item For easiness, we refer to this category as ``Single dose''
        \end{itemize}
  \item Drug screening: different drug combinations and concentrations used for experimental study of drug efficacy, which may or may not result in the killing of surrounding non-tumor cells in the human body with potential side effects.
        \begin{itemize}
            \item For easiness, we refer to this category as ``Drug screened''
        \end{itemize}
\end{enumerate}

The 60 inner wells are divided into sections to provide these three type of tumor tissues shown in figure \ref{fig:Single dose wellplate} and figure \ref{fig:Drug screen wellplate}.
\begin{figure}[H]
  \centering
  \includegraphics[width=0.9\linewidth]{figures/singledose.png} % Adjust the width as needed
  \caption{Well plate setup for the single-dose experiment where the left half remains untreated and the right half is treated with a single drug concentration. This image was taken three days after drug application, i.e., on day 10.}
  \label{fig:Single dose wellplate}
\end{figure}

\begin{figure}[H]
  \centering
  \includegraphics[width=1.2\linewidth]{figures/drug_screened.png} % Adjust the width as needed
  \caption{Well plate setup for the drug screening experiment where the majority of tumor tissues are treated with different combinations of drug concentrations (multi-colored wells), while some are left untreated (white wells bounded by boxes).}
  \label{fig:Drug screen wellplate}
\end{figure}

\ref{fig:time} Illustrates the flow chart of time evolutoion of 3D tumor tissues.
The 3D tumor tissue models develop in the wellplate progressively from day 1 to day 7, reaching their maximum cancerous state by day 7, at which point the drug is 
administered. By day 10, the drug's effect on the cancerous tissue is expected to peak, as nutrient availability gradually decreases and the tumor begins to diminish. 
To isolate the drug's effects, changes in tumor tissue deterioration are assessed on day 3 post-drug administration (day 10),  in accordance with established medical 
protocols and previous research findings. 
\begin{figure}[H]
  \centering
  \includegraphics[scale=0.9]{figures/timed.pdf} 
  \caption{Illustrates the flow chart of time evolutoion of 3D tumor tissues.}
  \label{fig:time}
\end{figure}

\chapter{Motivation}\label{ch: Motivation}

We assess the efficacy of the drug by comparing the changes it induces in the untreated bright-field microscopy images over a period of time. 
The current methods to differentiate these changes involve studying the alterations from day 7 (after applying the drug) to Day 10. These changes are typically
observed in three main parameters: 

\begin{enumerate}
  \item Size/Area
  \item Circularity/Diameter/Perimeter
  \item Pixel intensity or color change
\end{enumerate}

These parameters serve as human-interpretable metrics for assessing the efficacy of the drug. However, there may be other hidden information or patterns within 
these bright-field microscopy images that are not human-interpretable. This potential can be explored using representation learning techniques. Additionally, this 
method can provide more standardization compared to manual assessment.
\let\cleardoublepage\clearpage
