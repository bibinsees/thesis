\chapter{Literature review}
AI is cool!

\begin{figure}[htbp]
    \centering
    \includegraphics[width=\linewidth]{figures/Contextual_Ret_preprocessing.png} % replace with the path to your image file
    \caption{This is the caption for the figure.\cite{anthropic2024contextual}}
    \label{fig:your_label} % optional: label for referencing the figure
\end{figure}

\paragraph{Paragraph 1} \lipsum[2]

\paragraph{Paragraph 2} \lipsum[2]

\paragraph{Paragraph 3} \lipsum[2]
\lipsum[]

\section{Introduction to Retrieval-Augmented Generation (RAG) Architecture}
\lipsum[1]

\subsection{RAG's purpose and importance for document retrieval}
\lipsum[2]

\subsection{Core components of RAG}
\lipsum[2]

\subsubsection{Pre-Retrieval: Document Preparation}
\subsubsection{Retrieval: Document retrieval}
\subsubsection{Post-Retrieval: Optimizing retrieved documents}
\subsubsection{Generation: Answer generation}

\section{Pre-Retrieval: Document Preparation}

Sample algorithm:

\vspace{0.25cm}
\begin{algorithm}[H]
	\KwData{$\mathbf{X}, ...$}
	...
	\tcp*[f]{cool annotation} \\
	...\\
	... \\
	\Return $\mathbf{Y}, ...$
	\caption{Cool algorithm \cite{goodfellow2016deep}}
	\label{alg:cool_algorithm}
\end{algorithm}
\vspace{0.25cm}

\section{Retrieval Techniques}
\subsection{Dense Retrieval}
\subsection{Sparse Retrieval}
\subsection{Hybrid Retrieval}
\subsection{Challenges in German Regulatory Retrieval}
\section{Post-Retrieval: Optimizing Results}
\subsection{Re-Ranking}
\subsection{Vocabulary and Named Entity Recognition (NER)}
\section{Generation: Answer Synthesis}
\subsection{Role of Large Language Models (LLMs)}
\subsection{Context Length and Chunking Impact}
\section{Advancements and Research Gaps in RAG for Regulatory Retrieval}
\subsection{Emerging RAG Techniques}

\subsubsection{Graph RAG}

\subsubsection{Multimodal RAG}

\subsection{Research Gap in German Regulatory Texts}
Hello!