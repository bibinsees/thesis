\chapter{Literature Review}\label{ch:Literature Review}

\textbf{Base neural network architecture for representation learning.} Learning visual representations of medical images, such as X-rays (radiographic images) and bright-field microscopy images, is crucial for medical image understanding. However, progress in this area has been hindered by the heterogeneity and complexity of subtle features in these images, especially when they don't have labels. Existing work often relies on fine-tuning weights transferred from ImageNet pretraining (Wang et al., 2017 \cite{8099852} ; Esteva et al., 2017 \cite{Esteva2017Dermatologist} ; Irvin et al., 2019 \cite{irvin2019chexpert} ), which is suboptimal due to the drastically different characteristics of medical images. 

To address these challenges, researchers have proposed various innovative approaches. ConVIRT \cite{zhang2022contrastive} offers an alternative unsupervised strategy for learning medical visual representations by exploiting naturally occurring paired descriptive text. This method introduces a new approach to pretraining medical image encoders using paired text data via a bidirectional contrastive objective between the two modalities. It is domain-agnostic and requires no additional expert input.  However, given the absence of specific paired text data for our image dataset, ConVIRT does not offer a solution tailored to our specific problem.

The contrastive loss used in ConVIRT is derived from the SimCLR \cite{chen2020simple} self supervised learning framework. SimCLR learns representations by maximizing agreement between differently augmented views of the same data example via a contrastive loss in the latent space. The framework consists of a neural network base encoder that extracts representation vectors from augmented data examples. The framework allows for various choices of network architecture without any constraints. The authors opt for simplicity and adopt ResNet \cite{he2015deepresiduallearningimage}, introducing a learnable nonlinear transformation between the representation and the contrastive loss to substantially improve the quality of the learned representations. However, these methods require careful treatment of negative pairs, typically relying on large batch sizes to retrieve them. Additionally, their performance is highly dependent on the choice of image augmentations. 

After the ranking strategies outlined in the methodologies chapter  \ref{ch:Methodology for Ranking} were applied, reasonable metrics needed to be considered to support the relative assessment of drug efficacy. We refer this as 'inference metric'. Cosine distance was selected as the first choice because the SimCLR loss is designed to produce higher similarity between similar images. In the study by \cite{Mualla2013ACD}, Euclidean distance was employed to identify nearby cells for unstained cell detection in bright-field microscope images. Notably, alternative metrics have been proven useful in related applications; for example, \cite{CrossZamirski2022LabelFree} employed the Pearson correlation coefficient to measure similarity between predicted images (or their extracted features) and ground truth images in the context of label-free Cell Painting, where five fluorescent Cell Painting channels were predicted from brightfield input. In the study \cite{Lee2018DeepHCS} titled \textit{DeepHCS: Bright-Field to Fluorescence Microscopy Image Conversion}, L1 distance (Mean Absolute Error, MAE) was used as a loss function to measure the pixel-wise error between predicted and ground truth fluorescence images, thereby ensuring accurate translation of brightfield images into synthetic fluorescence images. Furthermore, the study \cite{Todorov2023STrack} on bacterial cell tracking (assessing the overlap of cell regions between consecutive frames) utilized the Jaccard index to compare the performance of different tracking tools, including on bright-field images of \textit{P. protegens} and \textit{S. pneumoniae}. These findings collectively support the use of cosine distance, Euclidean distance, Pearson correlation coefficient, L1 distance, and the Jaccard index as reasonable metrics for the relative assessment of drug efficacy.

Due to the problem of limited data, the exploration of ranking strategies that do not require specific training has been deemed essential. Dimensionality reduction techniques, such as PCA, t-SNE, and UMAP, have therefore been selected. In \cite{keyes2020cancerprimer}, PCA, t-SNE, and UMAP were applied to high-dimensional cytometry datasets in the study of human cancer, specifically for the analysis of bone marrow aspirates from acute myeloid leukemia patients, with the aim of reducing the data to two or three dimensions for analysis and interpretation of cytometry data.

In another study \cite{melba:2024:020:woodland}, PCA, UMAP, and t-SNE were applied to reduce the dimensionality of bottleneck features from a Swin UNETR trained for liver segmentation on T1-weighted MRIs. It was found that either PCA or UMAP improved performance over average pooling for all models tested. Since studies have demonstrated that PCA, t-SNE, and UMAP perform well on medical images, further exploration of these techniques has been conducted to reduce the higher dimensionality of the image data to a single dimension for ranking purposes.