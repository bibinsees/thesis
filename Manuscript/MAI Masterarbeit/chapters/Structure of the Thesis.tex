\chapter{Structure of the Thesis}\label{ch:Structure of the Thesis}

The goal of this thesis is to leverage representation learning of bright-field microscopy images using SimCLR to develop a ranking/ordering scale 
(1 to \( n \)) for all images. 

In this chapter, I will explain the framework designed to address this problem. The task is divided into three main pipelines.

First, we learn latent representations from the images using SimCLR as a self-supervised learning (SSL) model. The details will be explained in the 
\textit{Methodology for SimCLR} chapter.

Next, we perform an intermediate evaluation to assess whether the features have been effectively learned. The intermediate evaluation aims to:

\begin{enumerate}
    \item Classify the images,
    \item Cluster the images, and
    \item Ensure that the same features are extracted for identical images, even after transformations such as flipping, rotation, blurring, or brightness changes,
     using a direct distance measure approach.
\end{enumerate}

The details of this step will be explained in the \textit{Methodology for Intermediate Evaluation} chapter.

Subsequently, we use the learned features to establish a ranking scale. To achieve this, the following methods are employed:

\begin{enumerate}
    \item Prediction model,
    \item K-means centroid approach, and
    \item Softmax approach.
\end{enumerate}

The details of these methods will be discussed in the \textit{Methodology for Ranking Strategy} chapter.

Finally, we address an important question: why do we need to learn feature vectors from the images? Could the ranking task perform better using the raw images 
directly instead of SimCLR features? To answer this, we conduct a comparative study to evaluate the performance of both approaches.

The details of this comparative study will be covered in the \textit{Methodology for Comparative Study} chapter.

The overall structure of the thesis is illustrated in Figure~\ref{fig:BigOutline}.

\begin{figure}[H]
  \centering
  \includegraphics[scale=0.46]{figures/BigOutline.drawio.png} 
  \caption{Overall framework of the thesis.}
  \label{fig:BigOutline}
\end{figure}
